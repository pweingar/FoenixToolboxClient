\section{Channel Functions}
The channel functions provide support for channel or stream based I/O devices.
Any time a device or program wants to work with data as a sequential stream of bytes or characters,
that device should be a channel device.
Examples of channel devices that the Toolbox supports are the console (screen and keyboard), the serial port,
MIDI devices, and files open on SD cards.
In the future, there may be other channel devices as well ({\it e.g.} network streams).
Some channel devices can provide support for higher level functionality.
As an example, the console channel device provides support for printing ANSI terminal codes to the screen
and for reading certain keypresses back as ANSI escape sequences (for instance, function key presses).

\subsection*{sys\_chan\_read\_b -- 0xFFE028}
Read a single character from the channel. Returns the character, or 0 if none are available.

\bigskip

\begin{tabular}{|l||l|} \hline
Prototype & \lstinline!short sys_chan_read_b(short channel)! \\ \hline
channel & the number of the channel \\ \hline
Returns & the value read (if negative, error) \\ \hline
\end{tabular}

\subsubsection*{Example: C}
\begin{lstlisting}
    // Read a byte from channel #0 (keyboard)
    short b = sys_chan_read_b(0);
    if (b >= 0) {
        // We have valid data from 0-255 in b
    }
\end{lstlisting}

\subsubsection*{Example: Assembler}
\begin{verbatim}
    lda #0                 ; Select channel #0
    jsl sys_chan_read_b    ; Read from the channel
    bit #$ffff             ; If negative...
    bmi error              ; Process an error

    ; We have valid data in A
\end{verbatim}

\subsection*{sys\_chan\_read -- 0xFFE02C}
Read bytes from a channel and fill a buffer with them, given the number of the channel and the size of the buffer. Returns the number of bytes read.

\bigskip

\begin{tabular}{|l||l|} \hline
Prototype & \lstinline!short sys_chan_read(short channel, unsigned char * buffer, short size)! \\ \hline
channel & the number of the channel \\ \hline
buffer & the buffer into which to copy the channel data \\ \hline
size & the size of the buffer. \\ \hline
Returns & number of bytes read, any negative number is an error code \\ \hline
\end{tabular}

\subsubsection*{Example: C}
\begin{lstlisting}
    char buffer[80];
    short n = sys_chan_read(0, buffer, 80);
    if (n >= 0) {
        // We correctly read n bytes into the buffer
	} else {
        // We have an error
    }
\end{lstlisting}

\subsubsection*{Example: Assembler}
\begin{verbatim}
    pei #80              ; Push the size of the buffer

    pei #`buffer         ; Push the address of the buffer
    pei #<>buffer
	
    lda #0               ; Select channel #0
	
    jsl sys_chan_read    ; Try to read the bytes from the channel
	
    ply                  ; Clean up the stack
    ply
    ply
	
    bit #$ffff           ; If result is negative...
    bmi error            ; Go to process the error
	
    sta n                ; Otherwise: save the number of bytes read
\end{verbatim}

\subsection*{sys\_chan\_readline -- 0xFFE030}
Read a line of text from a channel (terminated by a newline character or by the end of the buffer). Returns the number of bytes read.

\bigskip

\begin{tabular}{|l||l|} \hline
Prototype & \lstinline!short sys_chan_readline(short channel, unsigned char * buffer, short size)! \\ \hline
channel & the number of the channel \\ \hline
buffer & the buffer into which to copy the channel data \\ \hline
size & the size of the buffer \\ \hline
Returns & number of bytes read, any negative number is an error code \\ \hline
\end{tabular}

\subsubsection*{Example: C}
\begin{lstlisting}
    short c = ...; // The channel number
    unsigned char buffer[128];
    short n = sys_chan_read_line(c, buffer, 128);    
\end{lstlisting}

\subsubsection*{Example: Assembler}
\begin{verbatim}
    pei #128                ; Push the size of the buffer
    pei #`buffer            ; Push the pointer to the buffer
    pei #<>buffer
    lda c                   ; Set the channel number to read from
    jsl sys_chan_read_line  ; Attempt to read a line from the console

    ply                     ; Clean up the stack
    ply
    ply

    sta n                   ; Save the number of bytes read
\end{verbatim}

\subsection*{sys\_chan\_write\_b -- 0xFFE034}
Write a single byte to the channel.

\bigskip

\begin{tabular}{|l||l|} \hline
Prototype & \lstinline!short sys_chan_write_b(short channel, uint8_t b)! \\ \hline
channel & the number of the channel \\ \hline
b & the byte to write \\ \hline
Returns & 0 on success, a negative value on error \\ \hline
\end{tabular}

\subsubsection*{Example: C}
\begin{lstlisting}
    // Write `a' to the console
    short result = sys_chan_write_b(0, 'a');
\end{lstlisting}

\subsubsection*{Example: Assembler}
\begin{verbatim}
    pei #'a'                ; Push 'a' as the b parameter
    lda #0                  ; Select the console (channel #0)
    jsl sys_chan_write_b    ; Write the character to the console
    ply                     ; Clean up the stack
\end{verbatim}

\subsection*{sys\_chan\_write -- 0xFFE038}
Write bytes from a buffer to a channel, given the number of the channel and the size of the buffer. Returns the number of bytes written.

\bigskip

\begin{tabular}{|l||l|} \hline
Prototype & \lstinline!short sys_chan_write(short channel, const uint8_t * buffer, short size)! \\ \hline
channel & the number of the channel \\ \hline
buffer &  \\ \hline
size &  \\ \hline
Returns & number of bytes written, any negative number is an error code \\ \hline
\end{tabular}

\subsubsection*{Example: C}
\begin{lstlisting}
    char * message = 'Hello, world!\n';
    short n = sys_chan_write(0, message, strlen(message));
\end{lstlisting}

\subsubsection*{Example: Assembler}
\begin{verbatim}
    pei #15               ; Push the size of the buffer
    pei #`message         ; Push the pointer to the message
    pei #<>message
    lda #0                ; Select the console (channel #0)
    jsl sys_chan_write    ; Write the buffer to the console

    ply                   ; Clean up the stack
    ply
    ply
    
    ; ...
message:
    .null "Hello, world!", 13, 10
\end{verbatim}

\subsection*{sys\_chan\_status -- 0xFFE03C}
Gets the status of the channel. The meaning of the status bits is channel-specific, but four bits are recommended as standard:

\begin{itemize}
\item 0x01: The channel has reached the end of its data
\item 0x02: The channel has encountered an error
\item 0x04: The channel has data that can be read
\item 0x08: The channel can accept data
\end{itemize}

\bigskip

\begin{tabular}{|l||l|} \hline
Prototype & \lstinline!short sys_chan_status(short channel)! \\ \hline
channel & the number of the channel \\ \hline
Returns & the status of the device \\ \hline
\end{tabular}

\subsubsection*{Example: C}
\begin{lstlisting}
    // Check the status of the file_in channel
    short status = sys_chan_status(file_in);
    if (status & 0x01) {
        // We have reached end of file
    }
\end{lstlisting}

\subsubsection*{Example: Assembler}
\begin{verbatim}
    lda file_in
    jsl sys_chan_status
    and #$01
    beq have_data
    ; We have reached end of file
\end{verbatim}

\subsection*{sys\_chan\_flush -- 0xFFE040}
Ensure any pending writes to a channel are completed.

\bigskip

\begin{tabular}{|l||l|} \hline
Prototype & \lstinline!short sys_chan_flush(short channel)! \\ \hline
channel & the number of the channel \\ \hline
Returns & 0 on success, any negative number is an error code \\ \hline
\end{tabular}

\subsubsection*{Example: C}
\begin{lstlisting}
    short file_out = ...;        // Channel number
    sys_chan_flush(file_out);    // Flush the channel
\end{lstlisting}

\subsubsection*{Example: Assembler}
\begin{verbatim}
    lda file_out          ; Channel number
    jsl sys_chan_flush    ; Flush the channel
\end{verbatim}

\subsection*{sys\_chan\_seek -- 0xFFE044}
Set the position of the input/output cursor. This function may not be honored by a given channel as not all channels are ``seekable.''
In addition to the usual channel parameter, the function takes two other parameters:
\begin{itemize}
\item position: the new position for the cursor
\item base: whether the position is absolute (0), or relative to the current position (1).
\end{itemize}

\bigskip

\begin{tabular}{|l||l|} \hline
Prototype & \lstinline!short sys_chan_seek(short channel, long position, short base)! \\ \hline
channel & the number of the channel \\ \hline
position & the position of the cursor \\ \hline
base & whether the position is absolute or relative to the current position \\ \hline
Returns & 0 = success, a negative number is an error. \\ \hline
\end{tabular}

\subsubsection*{Example: C}
\begin{lstlisting}
    short c = ...; // The channel number
    sys_chan_seek(c, -10, 1); // Move the point back 10 bytes
\end{lstlisting}

\subsubsection*{Example: Assembler}
\begin{verbatim}
    pei #1               ; Move is relative
    pei #-10             ; Move back by 10 bytes
    lda c                ; Select the channel
    jsl sys_chan_seek    ; Move the channel cursor

    ply                  ; Clean up the stack
    ply
\end{verbatim}

\subsection*{sys\_chan\_ioctrl -- 0xFFE048}
Send a command to a channel.
The mapping of commands and their actions are channel-specific.
The return value is also channel and command-specific.
The \verb+buffer+ and \verb+size+ parameters provide additional data to the commands,
what exactly needs to go in them (if anything) is command-specific.
Some commands require data in the buffer, and others do not.

\bigskip

\begin{tabular}{|l||l|} \hline
Prototype & \lstinline!short sys_chan_ioctrl(short channel, short command, uint8_t * buffer, short size)! \\ \hline
channel & the number of the channel \\ \hline
command & the number of the command to send \\ \hline
buffer & pointer to bytes of additional data for the command \\ \hline
size & the size of the buffer \\ \hline
Returns & 0 on success, any negative number is an error code \\ \hline
\end{tabular}

\subsubsection*{Example: C}
\begin{lstlisting}
    short c = ...;                   // The channel number
    short cmd = ...;                 // The command
    short r = sys_chan_ioctrl(c, cmd, 0, 0); // Send simple command
\end{lstlisting}

\subsubsection*{Example: Assembler}
\begin{verbatim}
    pei #0                 ; Push 0 for the size
    pei #0                 ; Push the null pointer for the buffer
    pei #0
    lda cmd                ; Push the command
    pha
    lda c
    jsl sys_chan_ioctrl    ; Issue the command

    ply                    ; Clean up the stack
    ply
    ply
    ply
\end{verbatim}


\subsection*{sys\_chan\_open -- 0xFFE04C}
Open a channel device for reading or writing given: the number of the device, the path to the resource on the device (if any), and the access mode. The access mode is a bitfield:
\begin{itemize}
    \item 0x01: Open for reading
    \item 0x02: Open for writing
    \item 0x03: Open for reading and writing
\end{itemize}

\bigskip

\begin{tabular}{|l||l|} \hline
Prototype & \lstinline!short sys_chan_open(short dev, const char * path, short mode)! \\ \hline
dev & the device number to have a channel opened \\ \hline
path & a "path" describing how the device is to be open \\ \hline
mode & s the device to be read, written, both\\ \hline
Returns & the number of the channel opened, negative number on error \\ \hline
\end{tabular}

\subsubsection*{Example: C}
\begin{lstlisting}
    // Serial port: 9600bps, 8-data bits, 1 stop bit, no parity
    short chan = sys_chan_open(2, "9600,8,1,N", 3);    
\end{lstlisting}

\subsubsection*{Example: Assembler}
\begin{verbatim}
    pei #3               ; Mode: Read & Write
    pei #`path           ; Pointer to the path
    pei #<>path
    lda #2               ; Device #2 (UART)
    jsl sys_chan_open    ; Open the channel to the UART

    ply                  ; Clean up the stack
    ply
    ply

    ; ...

path:
    .null "9600,8,1,N"
\end{verbatim}


\subsection*{sys\_chan\_close -- 0xFFE050}
Close a channel that was previously open by \lstinline|sys_chan_open|.

\bigskip

\begin{tabular}{|l||l|} \hline
Prototype & \lstinline!short sys_chan_close(short chan)! \\ \hline
chan & the number of the channel to close \\ \hline
Returns & nothing useful \\ \hline
\end{tabular}

\subsubsection*{Example: C}
\begin{lstlisting}
    short c = ...;          // The channel number
    sys_chan_close(c);      // Close the channel
\end{lstlisting}

\subsubsection*{Example: Assembler}
\begin{verbatim}
    lda c               ; Get the channel number
    jsl sys_chan_close  ; Close the channel
\end{verbatim}

\subsection*{sys\_chan\_swap -- 0xFFE054}
Swaps two channels, given their IDs. If before the call, channel ID \verb+channel1+ refers to the file ``hello.txt'', and channel ID \verb+channel2+ is the console, then after the call, \verb+channel1+ is the console, and \verb+channel2+ is the open file ``hello.txt''. Any context for the channels is preserved (for instance, the position of the file cursor in an open file).

\bigskip

\begin{tabular}{|l||l|} \hline
Prototype & \lstinline!short sys_chan_swap(short channel1, short channel2)! \\ \hline
channel1 & the ID of one of the channels \\ \hline
channel2 & the ID of the other channel \\ \hline
Returns & 0 on success, any other number is an error \\ \hline
\end{tabular}

\subsection*{sys\_chan\_device -- 0xFFE058}
Given a channel ID (the only parameter), return the ID of the device associated with the channel. The channel must be open.

\bigskip

\begin{tabular}{|l||l|} \hline
Prototype & \lstinline!short sys_chan_device(short channel)! \\ \hline
channel & the ID of the channel to query \\ \hline
Returns & the ID of the device associated with the channel, negative number for error \\ \hline
\end{tabular}

