\section{Channel Functions}

\subsubsection*{Example: C}
\begin{lstlisting}
\end{lstlisting}

\subsubsection*{Example: Assembler}
\begin{verbatim}
\end{verbatim}


\subsection*{sys\_chan\_read\_b -- 0xFFE024}
Read a single character from the channel. Returns the character, or 0 if none are available.


\bigskip

\begin{tabular}{|l||l|} \hline
Prototype & \lstinline!short sys_chan_read_b(short channel)! \\ \hline
channel & the number of the channel \\ \hline
Returns & the value read (if negative, error) \\ \hline
\end{tabular}

\subsubsection*{Example: C}
\begin{lstlisting}
    // Read a byte from channel #0 (keyboard)
    short b = sys_chan_read_b(0);
    if (b >= 0) {
        // We have valid data from 0-255 in b
    }
\end{lstlisting}

\subsubsection*{Example: Assembler}
\begin{verbatim}
    lda #0                 ; Select channel #0
    jsl sys_chan_read_b    ; Read from the channel
    bit #$ffff             ; If negative...
    bmi error              ; Process an error

    ; We have valid data in A
\end{verbatim}

\subsection*{sys\_chan\_read -- 0xFFE028}
Read bytes from a channel and fill a buffer with them, given the number of the channel and the size of the buffer. Returns the number of bytes read.

\bigskip

\begin{tabular}{|l||l|} \hline
Prototype & \lstinline!short sys_chan_read(short channel, unsigned char * buffer, short size)! \\ \hline
channel & the number of the channel \\ \hline
buffer & the buffer into which to copy the channel data \\ \hline
size & the size of the buffer. \\ \hline
Returns & number of bytes read, any negative number is an error code \\ \hline
\end{tabular}

\subsubsection*{Example: C}
\begin{lstlisting}
    char buffer[80];
    short n = sys_chan_read(0, buffer, 80);
    if (n >= 0) {
        // We correctly read n bytes into the buffer
	} else {
        // We have an error
    }
\end{lstlisting}

\subsubsection*{Example: Assembler}
\begin{verbatim}
    pei #80              ; Push the size of the buffer

    pei #`buffer         ; Push the address of the buffer
    pei #<>buffer
	
    lda #0               ; Select channel #0
	
    jsl sys_chan_read    ; Try to read the bytes from the channel
	
    ply                  ; Clean up the stack
    ply
    ply
	
    bit #$ffff           ; If result is negative...
    bmi error            ; Go to process the error
	
    sta n                ; Otherwise: save the number of bytes read
\end{verbatim}

\subsection*{sys\_chan\_readline -- 0xFFE02C}
\begin{tabular}{|l||l|} \hline
Prototype & \lstinline!short sys_chan_readline(short channel, unsigned char * buffer, short size)! \\ \hline
channel & the number of the channel \\ \hline
buffer & the buffer into which to copy the channel data \\ \hline
size & the size of the buffer \\ \hline
Returns & number of bytes read, any negative number is an error code \\ \hline
\end{tabular}

\subsection*{sys\_chan\_write\_b -- 0xFFE030}
\begin{tabular}{|l||l|} \hline
Prototype & \lstinline!short sys_chan_write_b(short channel, uint8_t b)! \\ \hline
channel & the number of the channel \\ \hline
b & the byte to write \\ \hline
Returns & 0 on success, a negative value on error \\ \hline
\end{tabular}

\subsection*{sys\_chan\_write -- 0xFFE034}
\begin{tabular}{|l||l|} \hline
Prototype & \lstinline!short sys_chan_write(short channel, const uint8_t * buffer, short size)! \\ \hline
channel & the number of the channel \\ \hline
buffer &  \\ \hline
size &  \\ \hline
Returns & number of bytes written, any negative number is an error code \\ \hline
\end{tabular}

\subsection*{sys\_chan\_status -- 0xFFE038}
\begin{tabular}{|l||l|} \hline
Prototype & \lstinline!short sys_chan_status(short channel)! \\ \hline
channel & the number of the channel \\ \hline
Returns & the status of the device \\ \hline
\end{tabular}

\subsection*{sys\_chan\_flush -- 0xFFE03C}
\begin{tabular}{|l||l|} \hline
Prototype & \lstinline!short sys_chan_flush(short channel)! \\ \hline
channel & the number of the channel \\ \hline
Returns & 0 on success, any negative number is an error code \\ \hline
\end{tabular}

\subsection*{sys\_chan\_seek -- 0xFFE040}
\begin{tabular}{|l||l|} \hline
Prototype & \lstinline!short sys_chan_seek(short channel, long position, short base)! \\ \hline
channel & the number of the channel \\ \hline
position & the position of the cursor \\ \hline
base & whether the position is absolute or relative to the current position \\ \hline
Returns & 0 = success, a negative number is an error. \\ \hline
\end{tabular}

\subsection*{sys\_chan\_ioctrl -- 0xFFE044}
\begin{tabular}{|l||l|} \hline
Prototype & \lstinline!short sys_chan_ioctrl(short channel, short command, uint8_t * buffer, short size)! \\ \hline
channel & the number of the channel \\ \hline
command & the number of the command to send \\ \hline
buffer & pointer to bytes of additional data for the command \\ \hline
size & the size of the buffer \\ \hline
Returns & 0 on success, any negative number is an error code \\ \hline
\end{tabular}

\subsection*{sys\_chan\_open -- 0xFFE048}
\begin{tabular}{|l||l|} \hline
Prototype & \lstinline!short sys_chan_open(short dev, const char * path, short mode)! \\ \hline
dev & the device number to have a channel opened \\ \hline
path & a "path" describing how the device is to be open \\ \hline
mode & s the device to be read, written, both? (0x01 = READ flag, 0x02 = WRITE flag, 0x03 = READ and WRITE) \\ \hline
Returns & the number of the channel opened, negative number on error \\ \hline
\end{tabular}

\subsection*{sys\_chan\_close -- 0xFFE04C}
\begin{tabular}{|l||l|} \hline
Prototype & \lstinline!short sys_chan_close(short chan)! \\ \hline
chan & the number of the channel to close \\ \hline
Returns & nothing useful \\ \hline
\end{tabular}

\subsection*{sys\_chan\_swap -- 0xFFE050}
\begin{tabular}{|l||l|} \hline
Prototype & \lstinline!short sys_chan_swap(short channel1, short channel2)! \\ \hline
channel1 & the ID of one of the channels \\ \hline
channel2 & the ID of the other channel \\ \hline
Returns & 0 on success, any other number is an error \\ \hline
\end{tabular}

\subsection*{sys\_chan\_device -- 0xFFE054}
\begin{tabular}{|l||l|} \hline
Prototype & \lstinline!short sys_chan_device(short channel)! \\ \hline
channel & the ID of the channel to query \\ \hline
Returns & the ID of the device associated with the channel, negative number for error \\ \hline
\end{tabular}

