\section{General Functions}

\subsection*{\texttt{sys\_proc\_exit}}
\begin{tabular}{|l||l|} \hline
Prototype & \lstinline!void sys_proc_exit(short result)! \\ \hline
Address & \texttt{0xFFE000} \\ \hline
\lstinline!result! & the code to return to the kernel \\ \hline
\end{tabular}

\subsection*{\texttt{sys\_proc\_run}}
\begin{tabular}{|l||l|} \hline
Prototype & \lstinline!short sys_proc_run(const char * path, int argc, char * argv[])! \\ \hline
Address & \texttt{0xFFE0D8} \\ \hline
\lstinline!path! & the path to the executable file \\ \hline
\lstinline!argc! & the number of arguments passed \\ \hline
\lstinline!argv! & the array of string arguments \\ \hline
Returns & the return result of the program \\ \hline
\end{tabular}

\subsection*{\texttt{sys\_text\_setsizes}}
\begin{tabular}{|l||l|} \hline
Prototype & \lstinline!void sys_text_setsizes(short chan)! \\ \hline
Address & \texttt{0x000000} \\ \hline
\lstinline!chan! &  \\ \hline
\end{tabular}

\subsection*{\texttt{sys\_mem\_get\_ramtop}}
\begin{tabular}{|l||l|} \hline
Prototype & \lstinline!uint32_t sys_mem_get_ramtop()! \\ \hline
Address & \texttt{0xFFE0B8} \\ \hline
Returns & the address of the first byte of reserved system RAM (one above the last byte the user program can use) \\ \hline
\end{tabular}

\subsection*{\texttt{sys\_mem\_reserve}}
\begin{tabular}{|l||l|} \hline
Prototype & \lstinline!uint32_t sys_mem_reserve(uint32_t bytes)! \\ \hline
Address & \texttt{0xFFE0BC} \\ \hline
\lstinline!bytes! & the number of bytes to reserve \\ \hline
Returns & address of the first byte of the reserved block \\ \hline
\end{tabular}

\subsection*{\texttt{sys\_time\_jiffies}}
\begin{tabular}{|l||l|} \hline
Prototype & \lstinline!uint32_t sys_time_jiffies()! \\ \hline
Address & \texttt{0xFFE0C0} \\ \hline
Returns & the number of jiffies since the last reset \\ \hline
\end{tabular}

\subsection*{\texttt{sys\_rtc\_set\_time}}
\begin{tabular}{|l||l|} \hline
Prototype & \lstinline!void sys_rtc_set_time(p_time time)! \\ \hline
Address & \texttt{0xFFE0C4} \\ \hline
\lstinline!time! & pointer to a t\_time record containing the correct time \\ \hline
\end{tabular}

\subsection*{\texttt{sys\_rtc\_get\_time}}
\begin{tabular}{|l||l|} \hline
Prototype & \lstinline!void sys_rtc_get_time(p_time time)! \\ \hline
Address & \texttt{0xFFE0C8} \\ \hline
\lstinline!time! & pointer to a t\_time record in which to put the current time \\ \hline
\end{tabular}

\subsection*{\texttt{sys\_kbd\_scancode}}
\begin{tabular}{|l||l|} \hline
Prototype & \lstinline!uint16_t sys_kbd_scancode()! \\ \hline
Address & \texttt{0xFFE0CC} \\ \hline
Returns & the next scan code from the keyboard... 0 if nothing pending \\ \hline
\end{tabular}

\subsection*{\texttt{sys\_kbd\_layout}}
\begin{tabular}{|l||l|} \hline
Prototype & \lstinline!short sys_kbd_layout(const char * tables)! \\ \hline
Address & \texttt{0xFFE0D4} \\ \hline
\lstinline!tables! & pointer to the keyboard translation tables \\ \hline
Returns & 0 on success, negative number on error \\ \hline
\end{tabular}

