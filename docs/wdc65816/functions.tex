\chapter{Toolbox Functions}

\section{Calling Convention}

All Toolbox functions are long call functions ({\it i.e.} using the \verb+JSL+ and \verb+RTL+ instructions) using the Calypsi ``simple call'' calling convention:
\begin{itemize}
	\item left-most parameter is placed in the accumulator for 8 and 16-bit types, and the X register and the accumulator for 24 and 32 bit types (X taking the most significant bits).
	\item remaining parameters are pushed on to the stack in right to left order (that is, the second parameter in a call is at the top of the stack just before the \verb+JSL+).
	\item 8-bit types are pushed as 16-bit values to avoid switching register sizes mid-call
	\item 24-bit types are pushed as 32-bit values for the same reason
	\item the return value is placed in the accumulator for 8 and 16-bit types, or in the X register and accumulator for 24 and 32 bit types (most significant bits in the X register).
	\item The caller is responsible for removing the parameters from the stack (if any) after the call returns.
\end{itemize}

Furthermore, Toolbox functions are written to save the direct page and data bank registers of the caller and to restore them before returning to the caller.
This means that a user program can do whatever it likes with the direct page and data bank registers,
and the Toolbox will not interfere with those settings. The Toolbox does use those registers itself,
but so long as the user program does not alter the Toolbox's RAM blocks (see the memory maps), there should be no interference between the two.

The Toolbox functions are accessed through a jump table located in the F256's flash memory, starting at 0xFFE000.
Each entry is four bytes long, and the address of each function is called out in their detailed descriptions below.

NOTE: Calypsi's ``simple call'' convention is not the fastest way to pass parameters to functions, and it is not Calypsi's only calling convention.
There is also a calling convention that uses pseudo-registers in the direct page to pass parameters.
Unfortunately, the rules for which parameter goes where in direct page are rather involved.
While that convention is preferable when Calypsi is the only compiler involved, the Toolbox needs to allow for other development tools to be used.
The stack based convention is more likely to be supported by other compilers. So speed was traded for broader compatibility.

\section{General Functions}

\subsection*{sys\_proc\_exit -- 0xFFE000}
This function ends the currently running program and returns control to the kernel. It takes a single short argument,
which is the result code that should be passed back to the kernel. This function does not return.

\bigskip

\begin{tabular}{|l|l|} \hline
\multicolumn{2}{|l|}{\lstinline!void sys_proc_exit(short result)!} \\ \hline\hline
result    & the code to return to the kernel \\ \hline
\end{tabular}

\subsubsection*{Example: C}
\begin{lstlisting}
sys_proc_exit(0);     // Quit the program with a result code of 0
\end{lstlisting}

\subsubsection*{Example: Assembler}
\begin{verbatim}
    lda #0                ; Return code of 0
    jsl sys_proc_exit     ; Quit the program
\end{verbatim}

\subsection*{sys\_proc\_run -- 0xFFE0DC}
Load and run an executable binary file.
This function will not return on success, since Foenix Toolbox is single tasking.
Any return value will be an error condition.

\bigskip

\begin{tabular}{|l||l|} \hline
Prototype & \lstinline!short sys_proc_run(const char * path, int argc, char * argv[])! \\ \hline
path & the path to the executable file \\ \hline
argc & the number of arguments passed \\ \hline
argv & the array of string arguments \\ \hline
Returns & the return result of the program \\ \hline
\end{tabular}

\subsubsection*{Example: C}
\begin{lstlisting}
    // Attempt to load and run /sd0/hello.pgx
    // Pass the command name and "test" as the arguments

    int argc = 2;
    char * argv[] = {
       "hello.pgx",
       "test"
    };
    short result = sys_proc_run("/sd0/hello.pgx", argc, argv);
\end{lstlisting}

\subsubsection*{Example: Assembler}
\begin{verbatim}
    pei #`argv          ; Push pointer to the arguments
    pei #<>argv
    pei #2              ; Push the argument count
    ldx #`path          ; Point to the path to load
    lda #<>path
    jsl sys_proc_run    ; Try to load and run the file

    ply                 ; Clean up the stack
    ply
    ply

    ; If we get here, there was an error loading or running
    ; the file. Error number is in the accumulator

    ...

path:
    .null "/sd0/hello.pgx"

argv:
    .null "hello.pgx"
    .null "test"
\end{verbatim}

\subsection*{sys\_proc\_set\_shell -- 0xFFE128 -- v1.01}
Set the address of a handler to be called in the event that a program calls \lstinline|sys_proc_exit|.
If the address is 0 (the default), the Toolbox will restart the machine when \lstinline|sys_proc_exit| is called.
If a non-zero address is provided, then the code at that address will be called in the same manner that program code
is started.
This function is provided to allow for the creation of shell programs and is not expected to be called by normal programs.

\bigskip

\begin{table}[!h]\begin{tabular}{|l||l|} \hline
Prototype & \lstinline!void sys_proc_set_shell(uint32_t address)! \\ \hline
Purpose & Set the address of the code that should handle a process exiting \\ \hline
address & the address of the handler code for proc\_exit \\ \hline
\end{tabular}\end{table}

\subsubsection*{Example: C}
\begin{lstlisting}
	uint32_t shell_entry = ...;
	sys_proc_set_shell(shell_entry);
\end{lstlisting}

\subsubsection*{Example: Assembler}
\begin{verbatim}
	ldx ##.word2 shell_entry
	lda ##.word0 shell_entry
	jsl sys_proc_set_shell
\end{verbatim}


\subsection*{sys\_proc\_get\_result -- 0xFFE12C -- v1.01}
If a program called \lstinline|sys_proc_exit|, this function returns the result code passed in that call.

\begin{table}[!h]\begin{tabular}{|l||l|} \hline
Prototype & \lstinline!int sys_proc_get_result()! \\ \hline
Purpose & Set the address of the code that should handle a process exiting \\ \hline
\end{tabular}\end{table}

\subsubsection*{Example: C}
\begin{lstlisting}
	// Get the result of the last program
	int result = sys_proc_get_result();	
\end{lstlisting}

\subsubsection*{Example: Assembler}
\begin{verbatim}
	; Get the result of the last program
	jsl sys_proc_get_result
	; Result code in the 16-bit accumulator
\end{verbatim}


\subsection*{sys\_reboot -- 0xFFE124 -- v1.01}
Force the system to reboot.

\bigskip

\begin{table}[!h]\begin{tabular}{|l||l|} \hline
Prototype & \lstinline!void sys_reboot()! \\ \hline
Purpose & Force the system to reboot \\ \hline
\end{tabular}\end{table}

\subsubsection*{Example: C}
\begin{lstlisting}
	// Reboot the F256
    sys_reboot();	
\end{lstlisting}

\subsubsection*{Example: Assembler}
\begin{verbatim}
	; Reboot the F256
    jsl sys_reboot
\end{verbatim}

\subsection*{sys\_get\_info -- 0xFFE020}
Fill out a structure with information about the computer. This information includes the model, the CPU, the amount of memory,
versions of the board and FPGAs, and what optional equipment is installed.

There is no return value.

\bigskip

\begin{tabular}{|l||l|} \hline
Prototype & \lstinline!void sys_get_info(p_sys_info info)! \\ \hline
info & pointer to a s\_sys\_info structure to fill out \\ \hline
\end{tabular}

\subsubsection*{Example: C}
\begin{lstlisting}
    struct s_sys_info info;
    sys_get_info(&info);
    printf("Machine: %s\n", info.model_name);	
\end{lstlisting}

\subsubsection*{Example: Assembler}
\begin{verbatim}
    ldx #`info			; Point to the info structure
    lda #<>info
    jsl sys_get_info

    ; The structure at info now has data in it
\end{verbatim}


\subsection*{sys\_mem\_get\_ramtop -- 0xFFE0BC}
Return the limit of accessible system RAM. The address returned is the first byte of memory that user programs may not access.
User programs may use any byte from the bottom of system RAM to RAMTOP - 1.

\bigskip

\begin{tabular}{|l||l|} \hline
Prototype & \lstinline!uint32_t sys_mem_get_ramtop()! \\ \hline
Returns & the address of the first byte of reserved system RAM \\ \hline
\end{tabular}


\subsection*{sys\_mem\_reserve -- 0xFFE0C0}
Reserve a block of memory from the top of system RAM.
This call will reduce the value returned by \lstinline|sys_get_ramtop| and will create a block of memory that user programs and the kernel will not change.
The current user program can load into that memory any code or data it needs to protect after it has quit
(for instance, a terminate-stay-resident code block). \lstinline|sys_mem_reserve| returns the address of the first byte of the block reserved.

NOTE: a reserved block cannot be returned to general use accept by restarting the system.

\bigskip

\begin{tabular}{|l||l|} \hline
Prototype & \lstinline!uint32_t sys_mem_reserve(uint32_t bytes)! \\ \hline
bytes & the number of bytes to reserve \\ \hline
Returns & address of the first byte of the reserved block \\ \hline
\end{tabular}

\subsubsection*{Example: C}
\begin{lstlisting}
    // Reserve a block of 256 bytes...
    uint32_t my_block = sys_mem_reserve(256);    
\end{lstlisting}

\subsubsection*{Example: Assembler}
\begin{verbatim}
    ldx #0                  ; Push the amount requested (256 bytes)
    lda #256
    jsl sys_mem_reserve     ; Attempt to reserve the block
    stx my_block+2          ; Save the address of the block reserved
    sta my_block
\end{verbatim}

\subsection*{sys\_time\_jiffies -- 0xFFE0C4}
Returns the number of ``jiffies'' since system startup.

A jiffy is 1/60 second. This clock counts the number of jiffies since the last system startup, but it is not terribly precise.
This counter should be sufficient for providing timeouts and wait delays on a fairly course level, but it should not be used when precision is required.

\bigskip

\begin{tabular}{|l||l|} \hline
Prototype & \lstinline!uint32_t sys_time_jiffies()! \\ \hline
Returns & the number of jiffies since the last reset \\ \hline
\end{tabular}

\subsubsection*{Example: C}
\begin{lstlisting}
    long jiffies = sys_time_jiffies();
\end{lstlisting}

\subsubsection*{Example: Assembler}
\begin{verbatim}
    jsl sys_time_jiffies    ; Get the time

    ; Jiffy count is now in X:A
\end{verbatim}

\subsection*{sys\_rtc\_set\_time -- 0xFFE0C8}
Sets the date and time in the real time clock. The date and time information is provided in an \verb+s_time+ structure (see below).

\bigskip

\begin{tabular}{|l||l|} \hline
Prototype & \lstinline!void sys_rtc_set_time(p_time time)! \\ \hline
time & pointer to a t\_time record containing the correct time \\ \hline
\end{tabular}

\subsubsection*{Example: C}
\begin{lstlisting}
    struct s_time time;
    
    // time structure is filled in with the current time

    // Set the time in the RTC
    sys_rtc_set_time(&time);
\end{lstlisting}

\subsubsection*{Example: Assembler}
\begin{verbatim}
    ; time structure is filled in with the current time

    ...

    ldx #`time              ; Point to the time structure
    lda #<>time
    jsl sys_rtc_set_time    ; Set the time in the RTC
\end{verbatim}


\subsection*{sys\_rtc\_get\_time -- 0xFFE0CC}
Gets the date and time in the real time clock. The date and time information is provided in an \verb+s_time+ structure (see below).

\bigskip

\begin{tabular}{|l||l|} \hline
Prototype & \lstinline!void sys_rtc_get_time(p_time time)! \\ \hline
time & pointer to a t\_time record in which to put the current time \\ \hline
\end{tabular}

\subsubsection*{Example: C}
\begin{lstlisting}
    struct s_time time;
    // ...
    sys_rtc_get_time(&time);
\end{lstlisting}

\subsubsection*{Example: Assembler}
\begin{verbatim}
    ldx #`time              ; Point to the time structure
    lda #<>time
    jsl sys_rtc_get_time    ; Get the time from the RTC
\end{verbatim}

\subsection*{sys\_kbd\_scancode -- 0xFFE0D0}
Returns the next keyboard scan code (0 if none are available).
Note that reading a scan code directly removes it from being used by the regular console code and may cause some surprising behavior if you combine the two.

See below for details about Foenix scan codes.

\bigskip

\begin{tabular}{|l||l|} \hline
Prototype & \lstinline!uint16_t sys_kbd_scancode()! \\ \hline
Returns & the next scan code from the keyboard... 0 if nothing pending \\ \hline
\end{tabular}

\subsubsection*{Example: C}
\begin{lstlisting}
    // Wait for a keypress
    uint16_t scan_code = 0;
    do {
        // Get the Foenix scan code from the keyboard
        scan_code = sys_kbd_scancode();
    } while (scan_code == 0);
\end{lstlisting}

\subsubsection*{Example: Assembler}
\begin{verbatim}
wait:
    jsl sys_kbd_scancode    ; Get the scan code from the keyboard
    cmp #0                  ; Keep checking until we get a keypress
    beq wait
\end{verbatim}


\subsection*{sys\_kbd\_layout -- 0xFFE0D8}
Sets the keyboard translation tables converting from scan codes to 8-bit character codes.
The table provided is copied by the kernel into its own area of memory, so the memory used in the calling program's memory space may be reused after this call.

Takes a pointer to the new translation tables (see below for details). If this pointer is 0, Foenix Toolbox will reset its translation tables to their defaults.

Returns 0 on success, or a negative number on failure.

\bigskip

\begin{tabular}{|l||l|} \hline
Prototype & \lstinline!short sys_kbd_layout(const char * tables)! \\ \hline
tables & pointer to the keyboard translation tables \\ \hline
Returns & 0 on success, negative number on error \\ \hline
\end{tabular}

\section{Channel Functions}
The channel functions provide support for channel or stream based I/O devices.
Any time a device or program wants to work with data as a sequential stream of bytes or characters,
that device should be a channel device.
Examples of channel devices that the Toolbox supports are the console (screen and keyboard), the serial port,
MIDI devices, and files open on SD cards.
In the future, there may be other channel devices as well ({\it e.g.} network streams).
Some channel devices can provide support for higher level functionality.
As an example, the console channel device provides support for printing ANSI terminal codes to the screen
and for reading certain key presses back as ANSI escape sequences (for instance, function key presses).

\subsection*{sys\_chan\_read\_b -- 0x000428}
Read a single character from the channel. Returns the character, or 0 if none are available.

\bigskip

\begin{tabular}{|l||l|} \hline
Prototype & \lstinline!short sys_chan_read_b(short channel)! \\ \hline
channel & the number of the channel \\ \hline
Returns & the value read (if negative, error) \\ \hline
\end{tabular}

\subsubsection*{Example: C}
\begin{lstlisting}
    // Read a byte from channel #0 (keyboard)
    short b = sys_chan_read_b(0);
    if (b >= 0) {
        // We have valid data from 0-255 in b
    }
\end{lstlisting}

\subsubsection*{Example: Assembler}
\begin{verbatim}
    clr.w d0               ; Select channel #0
    jsr sys_chan_read_b    ; Read from the channel
\end{verbatim}

\subsection*{sys\_chan\_read -- 0x00042C}
Read bytes from a channel and fill a buffer with them, given the number of the channel and the size of the buffer. Returns the number of bytes read.

\bigskip

\begin{tabular}{|l||l|} \hline
Prototype & \lstinline!short sys_chan_read(short channel, unsigned char * buffer, short size)! \\ \hline
channel & the number of the channel \\ \hline
buffer & the buffer into which to copy the channel data \\ \hline
size & the size of the buffer. \\ \hline
Returns & number of bytes read, any negative number is an error code \\ \hline
\end{tabular}

\subsubsection*{Example: C}
\begin{lstlisting}
    char buffer[80];
    short n = sys_chan_read(0, buffer, 80);
    if (n >= 0) {
        // We correctly read n bytes into the buffer
    } else {
        // We have an error
    }
\end{lstlisting}

\subsubsection*{Example: Assembler}
\begin{verbatim}
    move.w #80,-(a7)     ; Push the size of the buffer
    move.l #buffer,-(a7) ; Push the address of the buffer
    clr.w d0             ; Select channel #0
    
    jsr sys_chan_read    ; Try to read the bytes from the channel
    
    addq.l #6,a7         ; Clean up the stack
    
    btst #15,d0          ; If result is negative...
    beq error            ; Go to process the error
    
    move.w d0,n          ; Otherwise: save the number of bytes read
\end{verbatim}

\subsection*{sys\_chan\_readline -- 0x000430}
Read a line of text from a channel (terminated by a newline character or by the end of the buffer). Returns the number of bytes read.

\bigskip

\begin{tabular}{|l||l|} \hline
Prototype & \lstinline!short sys_chan_readline(short channel, unsigned char * buffer, short size)! \\ \hline
channel & the number of the channel \\ \hline
buffer & the buffer into which to copy the channel data \\ \hline
size & the size of the buffer \\ \hline
Returns & number of bytes read, any negative number is an error code \\ \hline
\end{tabular}

\subsubsection*{Example: C}
\begin{lstlisting}
    short c = ...; // The channel number
    unsigned char buffer[128];
    short n = sys_chan_read_line(c, buffer, 128);    
\end{lstlisting}

\subsubsection*{Example: Assembler}
\begin{verbatim}
    move.w #128,-(a7)       ; Push the size of the buffer
    move.l #buffer,-(a7)    ; Push the pointer to the buffer
    move.w c,d0             ; Set the channel number to read from

    jsr sys_chan_read_line  ; Attempt to read a line from the console

    addq.l #6,a7            ; Clean up the stack

    move.w d0,n             ; Save the number of bytes read
\end{verbatim}

\subsection*{sys\_chan\_write\_b -- 0x000434}
Write a single byte to the channel.

\bigskip

\begin{tabular}{|l||l|} \hline
Prototype & \lstinline!short sys_chan_write_b(short channel, uint8_t b)! \\ \hline
channel & the number of the channel \\ \hline
b & the byte to write \\ \hline
Returns & 0 on success, a negative value on error \\ \hline
\end{tabular}

\subsubsection*{Example: C}
\begin{lstlisting}
    // Write `a' to the console
    short result = sys_chan_write_b(0, 'a');
\end{lstlisting}

\subsubsection*{Example: Assembler}
\begin{verbatim}
    move.w 'a',-(a7)        ; Push 'a' as the b parameter
    clr.w d0                ; Select the console (channel #0)

    jsr sys_chan_write_b    ; Write the character to the console

    addq.l #2,a7            ; Clean up the stack
\end{verbatim}

\subsection*{sys\_chan\_write -- 0x000438}
Write bytes from a buffer to a channel, given the number of the channel and the size of the buffer. Returns the number of bytes written.

\bigskip

\begin{tabular}{|l||l|} \hline
Prototype & \lstinline!short sys_chan_write(short channel, const uint8_t * buffer, short size)! \\ \hline
channel & the number of the channel \\ \hline
buffer &  \\ \hline
size &  \\ \hline
Returns & number of bytes written, any negative number is an error code \\ \hline
\end{tabular}

\subsubsection*{Example: C}
\begin{lstlisting}
    char * message = 'Hello, world!\n';
    short n = sys_chan_write(0, message, strlen(message));
\end{lstlisting}

\subsubsection*{Example: Assembler}
\begin{verbatim}
    move.w #15,-(a7)      ; Push the size of the buffer
    move.l #message,-(a7) ; Push the pointer to the message
    move.w #0,d0          ; Select the console (channel #0)
    
    jsr sys_chan_write    ; Write the buffer to the console

    addq.l #6,a7          ; Clean up the stack
    
    ; ...
message:
    db.c "Hello, world!", 13, 10, 0
\end{verbatim}

\subsection*{sys\_chan\_status -- 0x00043C}
Gets the status of the channel. The meaning of the status bits is channel-specific, but four bits are recommended as standard:

\begin{itemize}
\item 0x01: The channel has reached the end of its data
\item 0x02: The channel has encountered an error
\item 0x04: The channel has data that can be read
\item 0x08: The channel can accept data
\end{itemize}

\bigskip

\begin{tabular}{|l||l|} \hline
Prototype & \lstinline!short sys_chan_status(short channel)! \\ \hline
channel & the number of the channel \\ \hline
Returns & the status of the device \\ \hline
\end{tabular}

\subsubsection*{Example: C}
\begin{lstlisting}
    // Check the status of the file_in channel
    short status = sys_chan_status(file_in);
    if (status & 0x01) {
        // We have reached end of file
    }
\end{lstlisting}

\subsubsection*{Example: Assembler}
\begin{verbatim}
    move.w file_in,d0

    jsr sys_chan_status

    andi.w $0001,d0
    beq have_data

    ; We have reached end of file
\end{verbatim}

\subsection*{sys\_chan\_flush -- 0x000440}
Ensure any pending writes to a channel are completed.

\bigskip

\begin{tabular}{|l||l|} \hline
Prototype & \lstinline!short sys_chan_flush(short channel)! \\ \hline
channel & the number of the channel \\ \hline
Returns & 0 on success, any negative number is an error code \\ \hline
\end{tabular}

\subsubsection*{Example: C}
\begin{lstlisting}
    short file_out = ...;        // Channel number
    sys_chan_flush(file_out);    // Flush the channel
\end{lstlisting}

\subsubsection*{Example: Assembler}
\begin{verbatim}
    move.w file_out,d0    ; Channel number
    jsr sys_chan_flush    ; Flush the channel
\end{verbatim}

\subsection*{sys\_chan\_seek -- 0x000444}
Set the position of the input/output cursor. This function may not be honored by a given channel as not all channels are ``seekable.''
In addition to the usual channel parameter, the function takes two other parameters:
\begin{itemize}
\item position: the new position for the cursor
\item base: whether the position is absolute (0), or relative to the current position (1).
\end{itemize}

\bigskip

\begin{tabular}{|l||l|} \hline
Prototype & \lstinline!short sys_chan_seek(short channel, long position, short base)! \\ \hline
channel & the number of the channel \\ \hline
position & the position of the cursor \\ \hline
base & whether the position is absolute or relative to the current position \\ \hline
Returns & 0 = success, a negative number is an error. \\ \hline
\end{tabular}

\subsubsection*{Example: C}
\begin{lstlisting}
    short c = ...; // The channel number
    sys_chan_seek(c, -10, 1); // Move the point back 10 bytes
\end{lstlisting}

\subsubsection*{Example: Assembler}
\begin{verbatim}
    move.w #1,-(a7)      ; Move is relative
    move.w #-10,-(a7)    ; Move back by 10 bytes
    move.w c,d0          ; Select the channel
    
    jsr sys_chan_seek    ; Move the channel cursor

    addq.l #4,a7         ; Clean up the stack
\end{verbatim}

\subsection*{sys\_chan\_ioctrl -- 0x000448}
Send a command to a channel.
The mapping of commands and their actions are channel-specific.
The return value is also channel and command-specific.
The \verb+buffer+ and \verb+size+ parameters provide additional data to the commands,
what exactly needs to go in them (if anything) is command-specific.
Some commands require data in the buffer, and others do not.

\bigskip

\begin{tabular}{|l||l|} \hline
Prototype & \lstinline!short sys_chan_ioctrl(short channel, short command, uint8_t * buffer, short size)! \\ \hline
channel & the number of the channel \\ \hline
command & the number of the command to send \\ \hline
buffer & pointer to bytes of additional data for the command \\ \hline
size & the size of the buffer \\ \hline
Returns & 0 on success, any negative number is an error code \\ \hline
\end{tabular}

\subsubsection*{Example: C}
\begin{lstlisting}
    short c = ...;                   // The channel number
    short cmd = ...;                 // The command
    short r = sys_chan_ioctrl(c, cmd, 0, 0); // Send simple command
\end{lstlisting}

\subsubsection*{Example: Assembler}
\begin{verbatim}
    move.w #0,-(a7)        ; Push 0 for the size
    move.l #0,-(a7)        ; Push the null pointer for the buffer
    move.w cmd,-(a7)       ; Push the command
    move.w c,d0            ; Select the channel

    jsr sys_chan_ioctrl    ; Issue the command

    add.l #8,a7           ; Clean up the stack

c   dc.b 0
cmd dc.w 0
\end{verbatim}


\subsection*{sys\_chan\_open -- 0x00044C}
Open a channel device for reading or writing given: the number of the device, the path to the resource on the device (if any), and the access mode.
The access mode is a bit field:
\begin{itemize}
    \item 0x01: Open for reading
    \item 0x02: Open for writing
    \item 0x03: Open for reading and writing
\end{itemize}

\bigskip

\begin{tabular}{|l||l|} \hline
Prototype & \lstinline!short sys_chan_open(short dev, const char * path, short mode)! \\ \hline
dev & the device number to have a channel opened \\ \hline
path & a "path" describing how the device is to be open \\ \hline
mode & s the device to be read, written, both\\ \hline
Returns & the number of the channel opened, negative number on error \\ \hline
\end{tabular}

\subsubsection*{Example: C}
\begin{lstlisting}
    // Serial port: 9600bps, 8-data bits, 1 stop bit, no parity
    short chan = sys_chan_open(2, "9600,8,1,N", 3);    
\end{lstlisting}

\subsubsection*{Example: Assembler}
\begin{verbatim}
    move.w #3,-(a7)      ; Mode: Read & Write
    move.l #path,-(a7)   ; Pointer to the path
    move.w #2,d0         ; Device #2 (UART)
    
    jsr sys_chan_open    ; Open the channel to the UART

    addq.l #6,a7         ; Clean up the stack

    ; ...

path:
    dc.b '9600,8,1,N',0
\end{verbatim}


\subsection*{sys\_chan\_close -- 0x000450}
Close a channel that was previously open by \lstinline|sys_chan_open|.

\bigskip

\begin{tabular}{|l||l|} \hline
Prototype & \lstinline!short sys_chan_close(short chan)! \\ \hline
chan & the number of the channel to close \\ \hline
Returns & nothing useful \\ \hline
\end{tabular}

\subsubsection*{Example: C}
\begin{lstlisting}
    short c = ...;          // The channel number
    sys_chan_close(c);      // Close the channel
\end{lstlisting}

\subsubsection*{Example: Assembler}
\begin{verbatim}
    move.w c,d0         ; Get the channel number
	
    jsr sys_chan_close  ; Close the channel
\end{verbatim}

\subsection*{sys\_chan\_swap -- 0x000454}
Swaps two channels, given their IDs. If before the call, channel ID \verb+channel1+ refers to the file ``hello.txt'',
and channel ID \verb+channel2+ is the console, then after the call, \verb+channel1+ is the console, and \verb+channel2+ is
the open file ``hello.txt''. Any context for the channels is preserved (for instance, the position of the file cursor in an open file).

\bigskip

\begin{tabular}{|l||l|} \hline
Prototype & \lstinline!short sys_chan_swap(short channel1, short channel2)! \\ \hline
channel1 & the ID of one of the channels \\ \hline
channel2 & the ID of the other channel \\ \hline
Returns & 0 on success, any other number is an error \\ \hline
\end{tabular}

\subsection*{sys\_chan\_device -- 0x000458}
Given a channel ID (the only parameter), return the ID of the device associated with the channel. The channel must be open.

\bigskip

\begin{tabular}{|l||l|} \hline
Prototype & \lstinline!short sys_chan_device(short channel)! \\ \hline
channel & the ID of the channel to query \\ \hline
Returns & the ID of the device associated with the channel, negative number for error \\ \hline
\end{tabular}


\section{Block Device Functions}
The block device functions provide low-level support for access to block-based storage devices. The main operations on block devices are reading a block of data from a device (given the device number and the address of the block to read), and writing a block of data to the device. These functions are used by the driver to the FatFS library to provide FAT32 file based access to those block devices.

Currently, the FA2560K2 supports two SD cards: device 0 is the external SD card, and 1 is the internal SD card. Future devices ({\it e.g.} the A2560M) may support other block devices like a floppy drive.

\subsection*{sys\_bdev\_register -- 0x000460}
Register a device driver for a block device. A device driver consists of a structure that specifies the name and number of the device as well as the various handler functions that implement the block device calls for that device.

See the section ``Extending the System'' below for more information.

\bigskip

\begin{tabular}{|l||l|} \hline
Prototype & \lstinline!short sys_bdev_register(p_dev_block device)! \\ \hline
device & pointer to the description of the device to register \\ \hline
Returns & 0 on success, negative number on error \\ \hline
\end{tabular}

\subsection*{sys\_bdev\_read -- 0x000464}
Read a block from a block device. Returns the number of bytes read.

\bigskip

\begin{tabular}{|l||l|} \hline
Prototype & \lstinline!short sys_bdev_read(short dev, long lba, uint8_t * buffer, short size)! \\ \hline
dev & the number of the device \\ \hline
lba & the logical block address of the block to read \\ \hline
buffer & the buffer into which to copy the block data \\ \hline
size & the size of the buffer. \\ \hline
Returns & number of bytes read, any negative number is an error code \\ \hline
\end{tabular}

\subsubsection*{Example: C}
\begin{lstlisting}
    unsigned char buffer[512];
    
    // Read the MBR of the interal SD card
    short n = sys_bdev_read(BDEV_SD1, 0, buffer, 512);
\end{lstlisting}

\subsubsection*{Example: Assembler}
\begin{verbatim}
    move.w #512,-(a7)   	; Push the size of the buffer
    move.l #buffer,-(a7)	; Push the pointer to the read buffer
    move.l #0,-(a7)     	; Push LBA = 0
    moveq.w #1,d0       	; The device number for the internal SD

    jsl sys_bdev_read   	; Read sector 0

    add.l #10,a7        	; Clean up the stack
\end{verbatim}

\subsection*{sys\_bdev\_write -- 0x000468}
Write a block from a block device. Returns the number of bytes written.

\bigskip

\begin{tabular}{|l||l|} \hline
Prototype & \lstinline!short sys_bdev_write(short dev, long lba, const uint8_t * buffer, short size)! \\ \hline
dev & the number of the device \\ \hline
lba & the logical block address of the block to write \\ \hline
buffer & the buffer containing the data to write \\ \hline
size & the size of the buffer. \\ \hline
Returns & number of bytes written, any negative number is an error code \\ \hline
\end{tabular}

\subsubsection*{Example: C}
\begin{lstlisting}
    unsigned char buffer[512];

    // Fill in the buffer with data...
    
    // Write the MBR of the interal SD card
    short n = sys_bdev_write(BDEV_SD1, 0, buffer, 512);
\end{lstlisting}

\subsubsection*{Example: Assembler}
\begin{verbatim}
    move.w #512,-(a7)   	; Push the size of the buffer
    move.l #buffer,-(a7)    ; Push the pointer to the read buffer
    move.l #0,-(a7)     	; Push LBA = 0
    moveq.w #1,d0       	; The device number for the internal SD

    jsl sys_bdev_write	; Write sector 0

    add.l #10,a7        ; Clean up the stack
\end{verbatim}

\subsection*{sys\_bdev\_status -- 0x00046C}
Gets the status of a block device. The meaning of the status bits is device specific, but there are two bits that are required in order to support the file system:
\begin{itemize}
    \item 0x01: Device has not been initialized yet
    \item 0x02: Device is present
\end{itemize}

\bigskip

\begin{tabular}{|l||l|} \hline
Prototype & \lstinline!short sys_bdev_status(short dev)! \\ \hline
dev & the number of the device \\ \hline
Returns & the status of the device \\ \hline
\end{tabular}

\subsubsection*{Example: C}
\begin{lstlisting}
    short bdev = ...; // The device number
    short status = sys_bdev_status(bdev);
\end{lstlisting}

\subsubsection*{Example: Assembler}
\begin{verbatim}
    move.w bdev,d0       ; Load the device number
    jsl sys_bdev_status  ; Attempt to get the status

    ; Status is in D0
\end{verbatim}


\subsection*{sys\_bdev\_flush -- 0x000470}
Ensure any pending writes to a block device are completed.

\bigskip

\begin{tabular}{|l||l|} \hline
Prototype & \lstinline!short sys_bdev_flush(short dev)! \\ \hline
dev & the number of the device \\ \hline
Returns & 0 on success, any negative number is an error code \\ \hline
\end{tabular}

\subsubsection*{Example: C}
\begin{lstlisting}
    short bdev = ...; // The device number
    sys_bdev_flush(bdev);
\end{lstlisting}

\subsubsection*{Example: Assembler}
\begin{verbatim}
    move.w bdev,d0      ; Load the device number
    jsl sys_bdev_flush  ; Attempt to flush pending writes
\end{verbatim}

\subsection*{sys\_bdev\_ioctrl -- 0x000474}
Send a command to a block device. The mapping of commands and their actions are device-specific. The return value is also device and command-specific.

Four commands should be supported by all devices:
\begin{itemize}
    \item \verb+GET_SECTOR_COUNT+ (1): Returns the number of physical sectors on the device
    \item \verb+GET_SECTOR_SIZE+ (2): Returns the size of a physical sector in bytes
    \item \verb+GET_BLOCK_SIZE+ (3): Returns the block size of the device. Really only relevant for flash devices and only needed by FatFS
    \item \verb+GET_DRIVE_INFO+ (4): Returns the identification of the drive
\end{itemize}

\bigskip

\begin{tabular}{|l||l|} \hline
Prototype & \lstinline!short sys_bdev_ioctrl(short dev, short command, uint8_t * buffer, short size)! \\ \hline
dev & the number of the device \\ \hline
command & the number of the command to send \\ \hline
buffer & pointer to bytes of additional data for the command \\ \hline
size & the size of the buffer \\ \hline
Returns & 0 on success, any negative number is an error code \\ \hline
\end{tabular}

\subsubsection*{Example: C}
\begin{lstlisting}
    short dev = ...;                // The device number
    short cmd = ...;                // The command
    short r = sys_bdev_ioctrl(dev, cmd, 0, 0); // Send simple command
\end{lstlisting}

\subsubsection*{Example: Assembler}
\begin{verbatim}
    move.w #0,-(a7)         ; Push buffer size of 0
    move.l #0,-(a7)         ; Push null pointer for buffer
    move.w cmd,-(a7)        ; Push the command number
    move.w bdev,d0          ; Select the block device
    
    jsl sys_bdev_ioctrl     ; Send the command

    add.l #8,a7            ; Clean up the stack
\end{verbatim}
\section{File System Functions}

% \subsubsection*{Example: C}
% \begin{lstlisting}
%     lorum ipsum
% \end{lstlisting}

% \subsubsection*{Example: Assembler}
% \begin{verbatim}
%     lorum ipsum
% \end{verbatim}

\subsection*{sys\_fsys\_open -- 0x000478}
Attempt to open a file in the file system for reading or writing.
Returns a channel number associated with the file.
If the returned number is negative, there was an error opening the file.

The mode parameter indicates how the file should be open and is a bit field, where each bit has a separate meaning:
\begin{itemize}
    \item 0x01: Read
    \item 0x02: Write
    \item 0x04: Create if new
    \item 0x08: Always create
    \item 0x10: Open file if it exists, otherwise create
    \item 0x20: Open file for appending
\end{itemize}

\bigskip		

\begin{tabular}{|l||l|} \hline
Prototype & \lstinline!short sys_fsys_open(const char * path, short mode)! \\ \hline
path & the ASCIIZ string containing the path to the file. \\ \hline
mode & the mode (e.g. r/w/create) \\ \hline
Returns & the channel ID for the open file (negative if error) \\ \hline
\end{tabular}

\subsubsection*{Example: C}
\begin{lstlisting}
    short chan = sys_fsys_open("hello.txt", 0x01);
    if (chan > 0) {
      // File is open for reading
    } else {
      // File was not open... chan has the error number
    }
\end{lstlisting}

\subsubsection*{Example: Assembler}
\begin{verbatim}
	move.w #1,-(a7)     ; Push the mode
	move.l #path,d0     ; Point to the path

    jsr sys_fsys_open   ; Try to open the file

	addq.l #2,a7        ; Clean up the stack

    btst.w #15,d0       ; Check to see if we opened the file
    beq error

    ; File is open for reading

error:

    ; There was an error
    ; The error number is in the accumulator

path:
    dc.b 'hello.txt',0
\end{verbatim}


\subsection*{sys\_fsys\_close -- 0x00047C}
Close a file that was previously opened, given its channel number.
If there were writes done on the channel, those writes will be committed to the block device holding the file.

\bigskip

\begin{tabular}{|l||l|} \hline
Prototype & \lstinline!short sys_fsys_close(short fd)! \\ \hline
fd & the channel ID for the file \\ \hline
Returns & 0 on success, negative number on failure \\ \hline
\end{tabular}

\subsubsection*{Example: C}
\begin{lstlisting}
    short chan = sys_fsys_open(...);
    // ...
    sys_fsys_close(chan);
\end{lstlisting}

\subsubsection*{Example: Assembler}
\begin{verbatim}
    move.w chan,d0
    jsr sys_fsys_close
\end{verbatim}


\subsection*{sys\_fsys\_opendir -- 0x000480}
Open a directory on a volume for reading, given its path.
Returns a directory handle number on success, or a negative number on failure.

\bigskip

\begin{tabular}{|l||l|} \hline
Prototype & \lstinline!short sys_fsys_opendir(const char * path)! \\ \hline
path & the path to the directory to open \\ \hline
Returns & the handle to the directory if >= 0. An error if < 0 \\ \hline
\end{tabular}

\subsubsection*{Example: C}
\begin{lstlisting}
    short dir = sys_fsys_opendir("/sd0/System");
    if (dir > 0) {
      // dir can be used for reading the directory entries
    } else {
      // There was an error... error number in dir
    }    
\end{lstlisting}

\subsubsection*{Example: Assembler}
\begin{verbatim}
	move.l #path,d0         ; Point to the path

    jsr sys_fsys_opendir    ; Try to open the directory

    btst.w #15,d0           ; Check to see if we opened the directory
    beq error

    ; Directory is open for reading

error:

    ; There was an error
    ; The error number is in the accumulator

path:
    dc.b '/sd0/System',0
\end{verbatim}


\subsection*{sys\_fsys\_closedir -- 0x000484}
Close a previously open directory, given its number.

\bigskip

\begin{tabular}{|l||l|} \hline
Prototype & \lstinline!short sys_fsys_closedir(short dir)! \\ \hline
dir & the directory handle to close \\ \hline
Returns & 0 on success, negative number on error \\ \hline
\end{tabular}

\subsubsection*{Example: C}
\begin{lstlisting}
    short dir = ... // Number of the directory to close
    sys_fsys_closedir(dir);
\end{lstlisting}

\subsubsection*{Example: Assembler}
\begin{verbatim}
    move.w dir,d0           ; Get the number of the directory to close
    jsr sys_fsys_closedir   ; Close the directory
\end{verbatim}


\subsection*{sys\_fsys\_readdir -- 0x000488}
Given the number of an open directory, and a buffer in which to place the data, fetch the file information of the next directory entry.
(See below for details on the \verb+file_info+ structure.)

Returns 0 on success, a negative number on failure.

\bigskip

\begin{tabular}{|l||l|} \hline
Prototype & \lstinline!short sys_fsys_readdir(short dir, p_file_info file)! \\ \hline
dir & the handle of the open directory \\ \hline
file & pointer to the t\_file\_info structure to fill out. \\ \hline
Returns & 0 on success, negative number on failure \\ \hline
\end{tabular}

\subsubsection*{Example: C}
\begin{lstlisting}
    short dir = sys_fsys_opendir("/sd0/System");
    if (dir > 0) {
      // dir can be used for reading the directory entries
      struct s_file_info file;
      if (sys_fsys_readdir(dir, &file_info) == 0) {
        // file_info contains information...
      } else {
        // Could not read the file entry...
      }
    } else {
      // There was an error... error number in dir
    }
\end{lstlisting}

\subsubsection*{Example: Assembler}
\begin{verbatim}
	move.l #path,d0         ; Point to the path

    jsr sys_fsys_opendir    ; Try to open the directory

    btst.w #15,d0           ; Check to see if we opened the directory
    beq error

    ; Directory is open for reading

	move.w d0, dir          ; Save the directory number

	move.l #file_info,-(a7) ; Set the pointer to the file info

    ; Directory number is already in D0

    jsr sys_fsys_readdir    ; Try to read from the directory

    addq.l 4,a7             ; Clean up the stack

    btst #15, d0            ; If result is <0, there is an error
    beq error

    ; Entry is loaded into structure at file_info

error:

    ; There was an error
    ; The error number is in the accumulator

path:
    dc.b '/sd0/System',0
file_info:
    ; ...
\end{verbatim}


\subsection*{sys\_fsys\_findfirst -- 0x00048C}
Given the path to a directory to search, a search pattern, and a pointer to a \verb+file_info+ structure,
return the first entry in the directory that matches the pattern.

Returns a directory handle on success, a negative number if there is an error

\bigskip

\begin{tabular}{|l||l|} \hline
Prototype & \lstinline!short sys_fsys_findfirst(const char * path, const char * pattern, p_file_info file)! \\ \hline
path & the path to the directory to search \\ \hline
pattern & the file name pattern to search for \\ \hline
file & pointer to the t\_file\_info structure to fill out \\ \hline
Returns & error if negative, otherwise the directory handle to use for subsequent calls \\ \hline
\end{tabular}

\subsubsection*{Example: C}
\begin{lstlisting}
    struct s_file_info file;
    short dir = sys_fsys_findfirst("/hd0/System/", "*.pgx", &file_info);
    if (dir == 0) {
      // file_info contains information...
    } else {
      // Could not read the file entry...
    }
\end{lstlisting}

\subsubsection*{Example: Assembler}
\begin{verbatim}
    move.l #file_info,-(a7) ; Point to the file_info 
    move.l #pattern,-(a7)   ; Point to the search pattern
    move.l #path,d0         ; Point to the directory to search

    jsr sys_fsys_findfirst  ; Try to find the first match

    addq.l #8,a7            ; Clean up the stack

    btst.w #15,d0           ; Check to see if error (negative)
    beq error

    ; File info should contain the first match

error:

    ; There was an error

file_info:
    ; ...
pattern:
    dc.b '*.pgx',0
path:
    dc.b '/sd0/System',0
\end{verbatim}


\subsection*{sys\_fsys\_findnext -- 0x000490}
Given the directory handle for a previously open search (from \verb+sys_fsys_findfirst+), and a \verb+file_info+ structure,
fill out the structure with the file information of the next file to match the original search pattern.

Returns 0 on success, a negative number if there is an error

\bigskip

\begin{tabular}{|l||l|} \hline
Prototype & \lstinline!short sys_fsys_findnext(short dir, p_file_info file)! \\ \hline
dir & the handle to the directory (returned by fsys\_findfirst) to search \\ \hline
file & pointer to the t\_file\_info structure to fill out \\ \hline
Returns & 0 on success, error if negative \\ \hline
\end{tabular}

\subsubsection*{Example: C}
\begin{lstlisting}
    struct s_file_info file;
    short dir = sys_fsys_findfirst("/hd0/System/", "*.pgx", &file_info);
    if (dir == 0) {
      // file_info contains information...

      // Look for the next...
      short result = sys_fsys_findnext(dir, &file_info);

    } else {
      // Could not read the file entry...
    }
\end{lstlisting}

\subsubsection*{Example: Assembler}
\begin{verbatim}
    move.l #file_info,-(a7) ; Point to the file_info 
    move.l #pattern,-(a7)   ; Point to the search pattern
    move.l #path,d0         ; Point to the directory to search

    jsr sys_fsys_findfirst  ; Try to find the first match

    addq.l #8,a7            ; Clean up the stack

    btst.w #15,d0           ; Check to see if error (negative)
    beq error

    move.w d0,dir           ; Save the open directory number

    ; File info should contain the first match

    ; ...

    ; Find the next

    move.l #file_info,-(a7) ; Point to the file_info
    move.w dir,d0           ; Get the directory number

    jsr sys_fsys_findnext   ; Try to find the next match

    addq.l #4,a7            ; Clean up the stack

    btst.w #15,d0           ; Check to see if error
    beq error

    ; File info should contain next match

error:

    ; There was an error

file_info:
    ; ...
pattern:
    dc.b '*.pgx',0
path:
    dc.b '/sd0/System',0
\end{verbatim}


\subsection*{sys\_fsys\_get\_label -- 0x000494}
Get the label of a volume.

\bigskip

\begin{tabular}{|l||l|} \hline
Prototype & \lstinline!short sys_fsys_get_label(const char * path, char * label)! \\ \hline
path & path to the drive \\ \hline
label & buffer that will hold the label... should be at least 35 bytes \\ \hline
Returns & 0 on success, error if negative \\ \hline
\end{tabular}

\subsubsection*{Example: C}
\begin{lstlisting}
    char label[64];
    short result = sys_fsys_get_label("/sd0", label);
\end{lstlisting}

\subsubsection*{Example: Assembler}
\begin{verbatim}
    move.l #label,-(a7)     ; Point to the label buffer
    move.l #path,d0         ; Point to the path of the drive

    jsr sys_fsys_get_label  ; Attempt to get the label

    addq.l #4,a7            ; Clean the stack

    btst.w #15,d0           ; Check for an error
    beq error

    ; We should have the label filled

error:

    ; There was an error

path:
    dc.b '/sd0',0
label:
    ds.b 64
\end{verbatim}


\subsection*{sys\_fsys\_set\_label -- 0x000498}
Set the label of a volume.

\bigskip

\begin{tabular}{|l||l|} \hline
Prototype & \lstinline!short sys_fsys_set_label(short drive, const char * label)! \\ \hline
drive & drive number \\ \hline
label & buffer that holds the label \\ \hline
Returns & 0 on success, error if negative \\ \hline
\end{tabular}

\subsubsection*{Example: C}
\begin{lstlisting}
    short result = sys_fsys_set_label(0, "FNXSD0");
\end{lstlisting}

\subsubsection*{Example: Assembler}
\begin{verbatim}
    move.l #label,-(a7)     ; Point to the label
    move.w #0,d0            ; Set the volume number

    jsr sys_fsys_set_label  ; Attempt to set the label

    addq.l #4,a7            ; Clean the stack

    btst.w #15,d0           ; Check for an error
    beq error

    ; We should have the label updated

error:

    ; There was an error

label:
    dc.b 'FNXSD0',0
\end{verbatim}

\subsection*{sys\_fsys\_mkdir -- 0x00049C}
Create a directory.

\bigskip

\begin{tabular}{|l||l|} \hline
Prototype & \lstinline!short sys_fsys_mkdir(const char * path)! \\ \hline
path & the path of the directory to create. \\ \hline
Returns & 0 on success, negative number on failure. \\ \hline
\end{tabular}

\subsubsection*{Example: C}
\begin{lstlisting}
    short result = sys_fsys_mkdir("/sd0/Samples");
\end{lstlisting}

\subsubsection*{Example: Assembler}
\begin{verbatim}
    move.l #path,-(a7)

    jsr sys_fsys_mkdir  ; Attempt to create the directory

    btst.w #15,d0       ; Check for an error
    beq error

    ; Directory should be created

error:

    ; There was an error

path:
    dc.b '/sd0/Samples',0
\end{verbatim}


\subsection*{sys\_fsys\_delete -- 0x0004A0}
Delete a file or directory, given its path. Returns 0 on success, a negative number if there is an error

\bigskip

\begin{tabular}{|l||l|} \hline
Prototype & \lstinline!short sys_fsys_delete(const char * path)! \\ \hline
path & the path of the file or directory to delete. \\ \hline
Returns & 0 on success, negative number on failure. \\ \hline
\end{tabular}

\subsubsection*{Example: C}
\begin{lstlisting}
    short result = sys_fsys_delete("/sd0/test.txt");
\end{lstlisting}

\subsubsection*{Example: Assembler}
\begin{verbatim}
    move.l #path,d0     ; Point to the path to delete

    jsr sys_fsys_delete ; Try to delete the file

    btst #15,d0
    beq error

    ; File was deleted...

error:

    ; There was an error

path:
    dc.b '/sd0/test.txt',0
\end{verbatim}


\subsection*{sys\_fsys\_rename -- 0x0004A4}
Rename a file or directory. Returns 0 on success, a negative number if there is an error

\bigskip

\begin{tabular}{|l||l|} \hline
Prototype & \lstinline!short sys_fsys_rename(const char * old_path, const char * new_path)! \\ \hline
old\_path & he current path to the file \\ \hline
new\_path & the new path for the file \\ \hline
Returns & 0 on success, negative number on failure. \\ \hline
\end{tabular}

\subsubsection*{Example: C}
\begin{lstlisting}
    short result = sys_fsys_rename("/sd0/test.txt", "doc.txt");
\end{lstlisting}

\subsubsection*{Example: Assembler}
\begin{verbatim}
    move.l #new_path,-(a7)	; Push the pointer to the new name
    move.l #old_path,d0     ; Point to the original file name

    jsr sys_fsys_rename 	; Try to rename the file the file

    addq.l #4,a7            ; Clean up the stack

    btst.w #15,d0           ; Check for an error
    beq error

    ; File was named...

error:

    ; There was an error

old_path:
    dc.b '/sd0/test.txt',0
new_path:
    dc.b 'doc.txt',0
\end{verbatim}

\subsection*{sys\_fsys\_load -- 0x0004A8}
Load a file into memory. This function can either load a file into a specific address provided by the caller,
or to the loading address specified in the file (for executable files). For executable files, the function will also
return the starting address specified in the file.

\bigskip

\begin{tabular}{|l||l|} \hline
Prototype & \lstinline!short sys_fsys_load(const char * path, uint32_t destination, uint32_t * start)! \\ \hline
path & the path to the file to load \\ \hline
destination & the destination address (0 for use file's address) \\ \hline
start & pointer to the long variable to fill with the starting address \\ \hline
Returns & 0 on success, negative number on error \\ \hline
\end{tabular}

\subsubsection*{Example: C}
\begin{lstlisting}
    uint32_t start;
    short result = sys_fsys_load("hello.pgx", 0, &start);
\end{lstlisting}

\subsubsection*{Example: Assembler}
\begin{verbatim}
    move.l #start,-(a7) ; Push the pointer to the start variable
    move.l #0,-(a7)     ; Push 0 to leave a load address unspecified
    move.l #path,d0     ; Point to the file name

    jsr sys_fsys_load   ; Try to rename the file the file

    addq.l #8,a7        ; Clean up the stack

    btst.w #15,d0       ; Check for an error
    beq error

    ; File was loaded

error:

    ; There was an error

path:
    dc.b 'hello.pgx',0
start:
    ds.l 1
\end{verbatim}


\subsection*{sys\_fsys\_register\_loader -- 0x0004AC}
Register a file loader for a binary file type.
A file loader is a function that takes a channel number for a file to load, a long representing the destination address,
and a pointer to a long for the start address of the program. These last two parameters are the same as are provided the \verb+sys_fsys_load+.

On success, returns 0. If there is an error in registering the loader, returns a negative number.

\bigskip

\begin{tabular}{|l||l|} \hline
Prototype & \lstinline!short sys_fsys_register_loader(const char * extension, p_file_loader loader)! \\ \hline
extension & the file extension to map to \\ \hline
loader & pointer to the file load routine to add \\ \hline
Returns & 0 on success, negative number on error \\ \hline
\end{tabular}

\subsubsection*{Example: C}
\begin{lstlisting}
    short foo_loader(short chan, uint32_t destination, uint32_t * start) {
        // Load file to destination (if provided)
        // If executable, set start to address to run
        return 0; // If successful
     };
     // ...
     short result = sys_fsys_register_loader("FOO", foo_loader);
\end{lstlisting}

\subsection*{sys\_fsys\_stat -- 0x0004B0}
Check to see if a file is present. The \verb+s_file_info+ structure will be populated if the file is found.
Returns 0 on success or a negative number on an error.

\bigskip

\begin{tabular}{|l||l|} \hline
Prototype & \lstinline!short sys_fsys_stat(const char * path, p_file_info file)! \\ \hline
path & the path to the file to check \\ \hline
file & pointer to a file info record to fill in, if the file is found. \\ \hline
Returns & 0 on success, negative number on error \\ \hline
\end{tabular}

\subsubsection*{Example: C}
\begin{lstlisting}
    s_file_info file_info;
    short result = sys_fsys_stat("/sd0/fnxboot.pgx", &file_info);
\end{lstlisting}
\section{Text System Functions}
Many programs will likely use the console channel device and the \verb+sys_chan_write+ call to print most things to the screen,
but there are certain operations that a program might need to carry out that do not fit well with the channel device. Also,
programs may want lower level control over the text screen. These functions are part of the text block of functions.

Functions in this block allow a program to find out what kinds of text modes the screen is capable of, change the
size of the display text, manipulate the cursor and the border of the screen, and even change the font and display colors.
Additionally, the text functions also provide for ``regions'' which may be used to create simple text windows---smaller rectangles
on the screen where printing will go, leaving other portions of the text screen unchanged.

The F256 supports only the one screen, but the text system functions were written with support for multiple screens in mind.
All text functions take a screen number. For the F256 as of the time of this writing, that number will always be 0.
If at some point, an F256 with multiscreen support is created or a graphics expansion card is produced, that additional
screen could be supported by the Toolbox with the addition of a text mode driver.

\subsection*{sys\_txt\_get\_capabilities  -- 0xFFE0E4}
Gets the description of a screen's capabilities.
The capabilities are returned as a pointer to a structure that provides a bit field of the various modes supported,
a listing of different font sizes that are supported (the F256 currently supports only 8 by 8 fonts),
and a listing of different screen resolutions supported by the screen.

\begin{lstlisting}
    struct s_txt_capabilities {
        short number;               /* The unique ID of the screen */
        short supported_modes;      /* The display modes supported on this screen */
        short font_size_count;      /* The number of supported font sizes */
        p_extent font_sizes;        /* Pointer to a list of t_extent listing all supported font sizes */
        short resolution_count;     /* The number of supported display resolutions */
        p_extent resolutions;       /* Pointer to a list of t_extent for supported resolutions (in pixels) */
    }
\end{lstlisting}

\bigskip

\begin{tabular}{|l||l|} \hline
Prototype & \lstinline!const p_txt_capabilities sys_txt_get_capabilities(short screen)! \\ \hline
screen & the number of the text device \\ \hline
Returns & a pointer to the read-only description (0 on error) \\ \hline
\end{tabular}

\subsection*{sys\_txt\_set\_mode -- 0x0004DC}
Set the display mode of the screen. There are five basic modes supported which are indicated by the five flags:
\begin{itemize}
    \item \verb+TXT_MODE_TEXT+---Render base text
    \item \verb+TXT_MODE_BITMAP+---Render bitmap graphics
    \item \verb+TXT_MODE_TILE+---Render tilesets
    \item \verb+TXT_MODE_SPRITE+---Render sprites
    \item \verb+TXT_MODE_SLEEP+---Puts the monitor in power-saving mode by turning off the sync signals
\end{itemize}

These flags are returned in the \verb+supported_modes+ field of the \verb+t_txt_capabilities+ structure returned by \verb+sys_txt_get_caps+,
and they may be combined to mix the different rendering engines if supported by the hardware
(for instance, \verb+TXT_MODE_TEXT | TXT_MODE_SPRITE+ would combine text and sprites). \verb+TXT_MODE_SLEEP+ will over-ride all the other modes.

The result of turning off all the mode flags is system dependent, but should result in a blank screen without putting the monitor into sleep mode.

Returns 0 on success, any other number means the mode was invalid for the screen or the screen was invalid.

\bigskip

\begin{tabular}{|l||l|} \hline
Prototype & \lstinline!short sys_txt_set_mode(short screen, short mode)! \\ \hline
screen & the number of the text device \\ \hline
mode & a bit field of desired display mode options \\ \hline
Returns & 0 on success, any other number means the mode is invalid for the screen \\ \hline
\end{tabular}

\subsubsection*{Example: C}
\begin{lstlisting}
    // Set screen 0 to text and tiles
    short result = sys_txt_set_mode(0, TXT_MODE_TEXT | TXT_MODE_TILE);
    if (result) {
      // Handle the error
    }    
\end{lstlisting}

\subsubsection*{Example: Assembler}
\begin{verbatim}
    move.w #(TXT_MODE_TEXT | TXT_MODE_TILE),-(a7)   ; Turn on text and tile modes
    move.w #0,d0                                	; On screen 0
    
    jsr sys_txt_set_mode

    addq.l #2,a7                                   	; Clean up the stack
\end{verbatim}

\subsection*{sys\_txt\_set\_resolution -- 0x0004E0}
Set the display resolution of the screen.
The width and height must match one of the resolutions listed in the screen's capabilities.

\bigskip

\begin{tabular}{|l||l|} \hline
Prototype & \lstinline!short sys_txt_set_resolution(short screen, short width, short height)! \\ \hline
screen & the number of the text device \\ \hline
width & the desired horizontal resolution in pixels \\ \hline
height & the desired vertical resolution in pixels \\ \hline
Returns & 0 on success, any other number means the mode is invalid for the screen \\ \hline
\end{tabular}

\subsubsection*{Example: C}
\begin{lstlisting}
    // Set screen 0 resolution to (320, 240)
    short result = sys_txt_set_resolution(0, 320, 240);
\end{lstlisting}

\subsubsection*{Example: Assembler}
\begin{verbatim}
    move.w #240,-(a7)        ; Resolution: 320 by 240
    move.w #320,-(a7)
    move.w #0,d0             ; On screen 0
    
    jsr sys_txt_set_mode

    addq.l #4,a7             ; Clean up the stack
\end{verbatim}


\subsection*{sys\_txt\_set\_xy -- 0x0004E8}
Sets the position of the cursor on the screen.

The call takes the number of the screen and the character row (y) and column (x) of the cursor.
The cursor positions are specified relative to the origin of the current region set on the screen, so (0, 0)
will be the origin of the region, (0, 1) will be the character position right below the origin, and so on.

\bigskip

\begin{tabular}{|l||l|} \hline
Prototype & \lstinline!void sys_txt_set_xy(short screen, short x, short y)! \\ \hline
screen & the number of the text device \\ \hline
x & the column for the cursor \\ \hline
y & the row for the cursor \\ \hline
\end{tabular}

\subsubsection*{Example: C}
\begin{lstlisting}
    // Move the cursor to the home position in the current region
    sys_txt_set_xy(0, 0, 0);
\end{lstlisting}

\subsubsection*{Example: Assembler}
\begin{verbatim}
    move.w #0,-(a7)         ; Set y = 0
    move.w #0,-(a7)         ; Set x = 0
    move.w #0,d0            ; Screen 0

    jsr sys_txt_set_xy      ; Set the position

    addq.l #4,a7            ; Clean up the stack
    ply
\end{verbatim}

\subsection*{sys\_txt\_get\_xy -- 0x0004EC}
Gets the position of the text cursor, given two parameters: the screen number, and the pointer to a \verb+t_point+.
The cursor position will be copied into the \verb+t_point+ object.

\bigskip

\begin{tabular}{|l||l|} \hline
Prototype & \lstinline!void sys_txt_get_xy(short screen, p_point position)! \\ \hline
screen & the number of the text device \\ \hline
position & pointer to a t\_point record to fill out \\ \hline
\end{tabular}

\subsubsection*{Example: C}
\begin{lstlisting}
    // Get the cursor position
    t_point position;
    sys_txt_get_xy(0, &position);
\end{lstlisting}

\subsubsection*{Example: Assembler}
\begin{verbatim}
    move.l #position,-(a7)  ; Pointer to the position object
    move.w #0,d0            ; Screen 0

    jsr sys_txt_get_xy      ; Get the cursor position

    addq.l #4,d0            ; Clean up the stack

position:
    ; ...
\end{verbatim}


\subsection*{sys\_txt\_get\_region -- 0x0004F0}
Gets the origin and size of the rectangle describing the current region.

The call takes a screen number and a pointer to a \verb+t_rect+ structure to fill out with the current information.
Returns 0 on success, any other number is an error.

\bigskip

\begin{tabular}{|l||l|} \hline
Prototype & \lstinline!short sys_txt_get_region(short screen, p_rect region)! \\ \hline
screen & the number of the text device \\ \hline
region & pointer to a t\_rect describing the rectangular region (using character cells for size and size) \\ \hline
Returns & 0 on success, any other number means the region was invalid \\ \hline
\end{tabular}

\subsubsection*{Example: C}
\begin{lstlisting}
    // Get the current region
    t_rect region;
    sys_txt_get_region(0, &region);
\end{lstlisting}

\subsubsection*{Example: Assembler}
\begin{verbatim}
    move.l #region,-(a7)    ; Pointer to the region object
    move.l #0,d0            ; Screen 0

    jsr sys_txt_get_region  ; Get the current region

    addq.l #4,a7            ; Clean up the stack

region:
    ; ...
\end{verbatim}


\subsection*{sys\_txt\_set\_region -- 0x0004F4}
Sets the rectangular region of the screen that will be used for all subsequent printing, scrolling, and filling.
This call takes the screen number and a pointer to a \verb+t_rect+ structure containing the origin (upper-left corner) and the size (width and height)
of the region. These values are specified in character cells, with (0, 0) being the upper-left corner of the screen.
If the size of the rectangle is 0 (width = height = 0), then the region will be the full screen.

Returns 0 on success, any other number is an error.

\bigskip

\begin{tabular}{|l||l|} \hline
Prototype & \lstinline!short sys_txt_set_region(short screen, p_rect region)! \\ \hline
screen & the number of the text device \\ \hline
region & pointer to a t\_rect describing the rectangular region (using character cells for size and size) \\ \hline
Returns & 0 on success, any other number means the region was invalid \\ \hline
\end{tabular}

\subsubsection*{Example: C}
\begin{lstlisting}
    // Set the region to a 5x5 panel in the upper left
    t_rect region;
    region.origin.x = 0;
    region.origin.y = 0;
    region.size.width = 5;
    region.size.height = 5;
    short result = sys_txt_set_region(0, &region);
    if (result) {
      // Handle the error
    }
\end{lstlisting}

\subsubsection*{Example: Assembler}
\begin{verbatim}
    move.l #region,-(a7)    ; Pointer to the region object
    move.w #0,d0            ; Screen 0

    jsr sys_txt_set_region  ; Set the new region

    addq.l #4,a7            ; Clean up the stack

region:
    dc.w 0, 0, 5, 5
\end{verbatim}

\subsection*{sys\_txt\_set\_color -- 0x0004F8}
Set the foreground and background color to use for subsequent prints to the screen.
Takes the screen number and the color indexes for foreground and background colors (0 -- 15).
Returns 0 on success, any other number is an error.

\bigskip

\begin{tabular}{|l||l|} \hline
Prototype & \lstinline!void sys_txt_set_color(short screen, unsigned char foreground, unsigned char background)! \\ \hline
screen & the number of the text device \\ \hline
foreground & the Text LUT index of the new current foreground color (0 -- 15) \\ \hline
background & the Text LUT index of the new current background color (0 -- 15) \\ \hline
\end{tabular}

\subsubsection*{Example: C}
\begin{lstlisting}
    // Set the text color to cyan on black (in standard colors)
    sys_txt_set_color(0, 6, 0);
\end{lstlisting}

\subsubsection*{Example: Assembler}
\begin{verbatim}
    move.w #0,-(a7)         ; Background to black
    move.w #6,-(a7)         ; Foreground to cyan
    move.w #0,d0            ; Screen 0

    jsr sys_txt_set_color   ; Set the text color

    addq.l #4,a7            ; Clean up the stack
\end{verbatim}


\subsection*{sys\_txt\_get\_color -- 0x0004FC}
Gets the current foreground and background color settings.
Takes the screen number and two pointers: one for the foreground color value, and one for the background color value.
Returns 0 on success, any other number is an error.

\bigskip

\begin{tabular}{|l||l|} \hline
Prototype & \lstinline!void sys_txt_get_color(short screen, unsigned char * foreground, unsigned char * background)! \\ \hline
screen & the number of the text device \\ \hline
foreground & the Text LUT index of the new current foreground color (0 - 15) \\ \hline
background & the Text LUT index of the new current background color (0 - 15) \\ \hline
\end{tabular}

\subsubsection*{Example: C}
\begin{lstlisting}
    // Gets the text color for the screen
    short foreground = 0;
    short background = 0;
    if (sys_txt_get_color(0, &foreground, &background)) {
      // Handle error
    }
\end{lstlisting}

\subsubsection*{Example: Assembler}
\begin{verbatim}
    move.l #background,-(a7)    ; Push address of background variable
    move.l #foreground,-(a7)    ; Push address of foreground variable
    move.w #0,d0                ; Screen 0

    jsr sys_txt_get_color       ; Get the color

    addq.l #8,a7                ; Clean up the stack

    ; ...

foreground:
    ds.w 1
background:
    ds.w 1
\end{verbatim}

\subsection*{sys\_txt\_set\_cursor -- 0x000500}
Set the appearance of the text mode cursor.

\bigskip

\begin{tabular}{|l||l|} \hline
Prototype & \lstinline!void sys_txt_set_cursor(short screen, short enable, short rate, char c)! \\ \hline
screen & the screen number \\ \hline
enable & 0 to hide, any other number to make visible \\ \hline
rate & the blink rate for the cursor (0=1s, 1=0.5s, 2=0.25s, 3=1/5s) \\ \hline
char & the character in the current font to use as a cursor \\ \hline
\end{tabular}

\subsubsection*{Example: C}
\begin{lstlisting}
    // Set the cursor on screen 0
    // Visible, blink period 0.25s, character @
    sys_txt_set_cursor(0, 1, 2, '@');
\end{lstlisting}

\subsubsection*{Example: Assembler}
\begin{verbatim}
    move.w #'@',-(a7)   ; Cursor character @
    move.w #2,-(a7)     ; Blink period 0.25s
    move.w #1,-(a7)     ; Show the cursor
    move.w #0,d0        ; Screen 0

    jsr sys_txt_set_cursor

    addq.l #6,a7        ; Clean the stack
\end{verbatim}


\subsection*{sys\_txt\_set\_cursor\_visible -- 0x000504}
Sets the visibility of the text cursor.

\bigskip

\begin{tabular}{|l||l|} \hline
Prototype & \lstinline!void sys_txt_set_cursor_visible(short screen, short is_visible)! \\ \hline
screen & the screen number \\ \hline
is\_visible & TRUE if the cursor should be visible, FALSE (0) otherwise \\ \hline
\end{tabular}

\subsubsection*{Example: C}
\begin{lstlisting}
    // Hide the cursor on screen 0
    sys_txt_set_cursor_visible(0, 0);
\end{lstlisting}

\subsubsection*{Example: Assembler}
\begin{verbatim}
    move.w #0,-(a7) ; Hide the cursor
    move.w #0,d0    ; Screen 0

    jsr sys_txt_set_cursor_visible

    addq.l #2,a7    ; Clean the stack
\end{verbatim}


\subsection*{sys\_txt\_set\_font -- 0x000508}
Set the font to be used in text mode on the screen. Takes the screen number, the width and height of the characters (in pixels),
and a pointer to the actual font data. Returns 0 on success, any other number means the screen is invalid, or the font size is invalid.

NOTE: the font size must be listed in the \verb+font_sizes field+ of the \verb+t_txt_capabilities+ structure returned by \verb+sys_txt_get_caps+.

\bigskip

\begin{tabular}{|l||l|} \hline
Prototype & \lstinline!short sys_txt_set_font(short screen, short width, short height, unsigned char * data)! \\ \hline
screen & the number of the text device \\ \hline
width & width of a character in pixels \\ \hline
height & of a character in pixels \\ \hline
data & pointer to the raw font data to be loaded \\ \hline
\end{tabular}

\subsubsection*{Example: C}
\begin{lstlisting}
    // Set the font of screen 0 to an 8x8 font
    unsigned char * font_data;
    font_data = ...;
    short result = sys_txt_set_font(0, 8, 8, font_data);
    if (result) {
      // Handle error
    }    
\end{lstlisting}

\subsubsection*{Example: Assembler}
\begin{verbatim}
    move.l #font_data,-(a7)     ; Push pointer to the font data
    move.w #8,-(a7)             ; Push size of 8x8
    move.w #0,d0                ; Screen 0

    jsr sys_txt_set_font        ; Set the font

    addq.l #6,a7                ; Clean up the stack
\end{verbatim}


\subsection*{sys\_txt\_setsizes -- 0x0004E4}
Sets the text screen device driver to the current screen geometry, based on the display resolution and border size.
If a program changes the border or display resolution on its own but still needs to use the Toolbox console or text routines to display text,
it should call this function to have the Toolbox recalculate the screen geometry.

\bigskip

\begin{tabular}{|l||l|} \hline
Prototype & \lstinline!void sys_txt_setsizes(short screen)! \\ \hline
screen & the number of the text device \\ \hline
\end{tabular}

\subsubsection*{Example: C}
\begin{lstlisting}
    // Recalculate geometry of screen 0
    sys_txt_setsizes(0);
\end{lstlisting}

\subsubsection*{Example: Assembler}
\begin{verbatim}
    move.l #0,d0
    jsr sys_txt_setsizes
\end{verbatim}

\subsection*{sys\_txt\_get\_sizes -- 0x00050C}
Gets the size of the screen in total pixels (not taking the border into consideration) and visible characters (taking the border into account).

NOTE: \verb+text_size+ and \verb+pixel_size+ can be null (0), in which case that structure will not be filled out,
so you do not have to provide a \verb+t_extent+ for a measurement you do not need.

\begin{tabular}{|l||l|} \hline
Prototype & \lstinline!void sys_txt_get_sizes(short screen, p_extent text_size, p_extent pixel_size)! \\ \hline
screen & the screen number  \\ \hline
text\_size & the size of the screen in visible characters (may be null) \\ \hline
pixel\_size & the size of the screen in pixels (may be null) \\ \hline
\end{tabular}

\subsubsection*{Example: C}
\begin{lstlisting}
    // Hide the cursor on screen 0
    t_rect text_matrix;
    t_rect pixel_matrix;
    sys_txt_get_sizes(0, &text_matrix, &pixel_matrix);    
\end{lstlisting}

\subsubsection*{Example: Assembler}
\begin{verbatim}
    move.l #pixel_matrix,-(a7)  ; Push pointer to pixel extent
    move.l #text_matrix,-(a7)   ; Push pointer to text extent
    move.w #0,d0

    jsr sys_txt_get_sizes       ; Get the sizes

    addq.l #8,a7                ; Clean up the stack

    ; ...

pixel_matrix:                   ; Holds size of screen in pixels
    ds.w 2
text_matrix:                    ; Holds size of screen in characters
    ds.w 2
\end{verbatim}


\subsection*{sys\_txt\_set\_border -- 0x000510}
Sets the size of the border around the screen. Takes the number of the screen and the size of the border width and height.
In this context, width is the width of the left and right borders taken separately, and height is the height of the top and bottom borders.
So if width is 8 and height is 16, 32 lines will be taken up by the top and bottom borders together,
and 16 columns will be taken up by the left and right borders.

NOTE: if the width and height of the borders are 0, the border will be disabled.

\bigskip

\begin{tabular}{|l||l|} \hline
Prototype & \lstinline!void sys_txt_set_border(short screen, short width, short height)! \\ \hline
screen & the number of the text device \\ \hline
width & the horizontal size of one side of the border (0 -- 32 pixels) \\ \hline
height & the vertical size of one side of the border (0 -- 32 pixels) \\ \hline
\end{tabular}

\subsubsection*{Example: C}
\begin{lstlisting}
    // Set the border on screen 0: width of 16, height of 8
    sys_txt_set_border(0, 16, 8);
\end{lstlisting}

\subsubsection*{Example: Assembler}
\begin{verbatim}
    move.w #8,-(a7)         ; 8 pixels vertically
    move.w #16,-(a7)        ; 16 pixels horizontally
    move.w #0,d0            ; Screen 0

    jsr sys_txt_set_border  ; Set the border size

    addq.l #4,a7            ; Clean up the stack
\end{verbatim}


\subsection*{sys\_txt\_set\_border\_color -- 0x000514}
Set the color of the border, using red, green, and blue components (which may go from 0 to 255).

\bigskip

\begin{tabular}{|l||l|} \hline
Prototype & \lstinline!void sys_txt_set_border_color(short screen, unsigned char red, unsigned char green, unsigned char blue)! \\ \hline
screen & the number of the text device \\ \hline
red & the red component of the color (0 - 255) \\ \hline
green & the green component of the color (0 - 255) \\ \hline
blue & the blue component of the color (0 - 255) \\ \hline
\end{tabular}

\subsubsection*{Example: C}
\begin{lstlisting}
    // Set the border of screen 0 to dark blue
    sys_txt_set_border_color(0, 0, 0, 128);
\end{lstlisting}

\subsubsection*{Example: Assembler}
\begin{verbatim}
    move.w #128,-(a7)           ; Push blue
    move.w #0,-(a7)             ; Push green
    move.w #0,-(a7)             ; Push red
    move.w #0,d0                ; Screen 0

    jsr sys_txt_set_border      ; Set the border color

    addq.l #6,a7                ; Clean up the stack
\end{verbatim}

\subsection*{sys\_txt\_put -- 0x000518}
Print a character to the screen.

NOTE: No this function does not interpret ANSI terminal codes and will display
the characters corresponding to those bytes on the screen. To print with ANSI
terminal code support, use the console channel device.

\bigskip

\begin{tabular}{|l||l|} \hline
Prototype & \lstinline!void sys_txt_put(short screen, char c)! \\ \hline
screen & the number of the text device \\ \hline
c & the character to print \\ \hline
\end{tabular}

\subsubsection*{Example: C}
\begin{lstlisting}
    // Print 'A' to the screen
    sys_txt_put(0, 'A');
\end{lstlisting}

\subsubsection*{Example: Assembler}
\begin{verbatim}
    move.w #'A',-(a7)   ; Push the character
    move.w #0,d0        ; Screen 0
    
    jsr sys_txt_put     ; Print the character

    addq.l #2,a7        ; Clean up the stack
\end{verbatim}

\subsection*{sys\_txt\_print -- 0x00051C}
Print a null-terminated ASCII string to the screen.

NOTE: No this function does not interpret ANSI terminal codes and will display
the characters corresponding to those bytes on the screen. To print with ANSI
terminal code support, use the console channel device.

\bigskip

\begin{tabular}{|l||l|} \hline
Prototype & \lstinline!void sys_txt_print(short screen, const char * message)! \\ \hline
screen & the number of the text device \\ \hline
message & the ASCII Z string to print \\ \hline
\end{tabular}

\subsubsection*{Example: C}
\begin{lstlisting}
    // Print a message to the screen
    sys_txt_print(0, "Hello, Foenix!\n");
\end{lstlisting}

\subsubsection*{Example: Assembler}
\begin{verbatim}
    move.w #message,-(a7)   ; Push pointer to message
    move.w #0,d0            ; Screen 0

    jsr sys_txt_print       ; Print the message

    addq.l #4,a7            ; Clean up the stack

    ; ...

message:
    dc.b "Hello, Foenix!",13,10,0
\end{verbatim}
\section{Interrupt Functions}
Interrupts in the Toolbox are managed at a device level.
The F256 includes an interrupt controller which assigns a different interrupt to each device that can raise an interrupt.
The interrupt controller provides for separate masking and interrupt flags for each device interrupt.
The Toolbox allows programs to register an interrupt handler for the specific device-level interrupt the program needs to handle.
That handler is just a regular subroutine (it should not be coded to return with an \verb+RTI+ instruction).
The Toolbox will take care of checking the various interrupt controller registers to determine which interrupts are currently pending
and will call the associated interrupt handler automatically. The Toolbox will also take care to save register states to avoid interfering
with the currently running program.

On the F256, interrupts are arranged into three groups, with each interrupt getting a bit within one of the groups in each of the 
control registers for that group. For instance, the start of frame interrupt (vertical blank interrupt) is the least significant bit of group 0.
The Foenix Toolbox, on the other hand, assigns a single number to each interrupt and internally maps that number to the appropriate group and bit.
When enabling or disabling an interrupt or when registering an interrupt handler, it is this internal interrupt number that is used.\footnote{The interrupt
number assignments may seem arbitrary, but they are actually just inherited from Foenix MCP and the A2560s, where the interrupt number is essentially just
the group and bit numbers packed into an 8-bit value.}

In addition to controlling interrupts at a device level ({\it e.g.} the serial port interrupt), the Toolbox also has routines to allow a program to
enable or disable IRQ processing at the CPU level. The \lstinline!int_enable_all!, \lstinline!int_disable_all!, and \lstinline!int_restore_all!,
functions work at the CPU level and do not affect the mask bits in the F256's interrupt controller. The \lstinline!int_enable!, and \lstinline!int_disable!
functions work at the level of the individual device interrupt in the interrupt controller. 
 
\subsection*{Overriding Toolbox Interrupt Handling}
The Toolbox takes care of the details of the F256's interrupt controller for user programs, but that comes at the cost of overhead.
Many programs will need more efficient control over interrupts and will prefer to manage interrupts for themselves, without the 
Toolbox intervening.

This is perfectly fine.

The 65816's interrupt vectors are stored in RAM (the F256 populates them based on data in flash at boot time).
Programs can write their own interrupt handler addresses to those vectors to take over the handling of interrupts.
The only problem with this is that the Toolbox depends upon interrupts to handle keystrokes.
By default, the Toolbox uses the start-of-frame (SOF) interrupt on the F256k and F256k2 to periodically trigger a
scan of the keyboard matrix and the PS/2 interrupt on the F256jr.
If a program takes over interrupts but still needs to use the Toolbox's keyboard routines, it will need to call
\lstinline!sys_kbd_handle_irq! function periodically (see below) to trigger the matrix scan or check the PS/2 port for keystrokes.

\begin{table}
    \begin{center}
        \begin{tabular}{|c|c||c|l|} \hline 
            \multicolumn{2}{|c||}{Hardware} & Toolbox & \\
            Group & Bit & Number & Purpose \\ \hline
            \multirow{8}{*}{0} & 0x01 & 0x00 & Start of frame \\ \cline{2-4}
                & 0x02 & 0x01 & Start of line \\ \cline{2-4}
                & 0x04 & 0x10 & PS/2 Keyboard \\ \cline{2-4}
                & 0x08 & 0x12 & PS/2 Mouse \\ \cline{2-4}
                & 0x10 & 0x18 & Timer 0 \\ \cline{2-4}
                & 0x20 & 0x19 & Timer 1 \\ \cline{2-4}
                & 0x40 & ---  & Reserved \\ \cline{2-4}
                & 0x80 & 0x06 & External Expansion \\ \hline\hline

            \multirow{8}{*}{1} & 0x01 & 0x13 & Serial Port \\ \cline{2-4}
                & 0x02 & ---  & Reserved \\ \cline{2-4}
                & 0x04 & ---  & Reserved \\ \cline{2-4}
                & 0x08 & ---  & Reserved \\ \cline{2-4}
                & 0x10 & 0x1F & Real Time Clock \\ \cline{2-4}
                & 0x20 & 0x1D & VIA 0 \\ \cline{2-4}
                & 0x40 & 0x1E & VIA 1 (F256K) \\ \cline{2-4}
                & 0x80 & 0x22 & SD Card \\ \hline\hline

            \multirow{8}{*}{2} & 0x01 & --- & Reserved \\ \cline{2-4}
                & 0x02 & --- & Reserved \\ \cline{2-4}
                & 0x04 & --- & Reserved \\ \cline{2-4}
                & 0x08 & --- & Reserved \\ \cline{2-4}
                & 0x10 & --- & Reserved \\ \cline{2-4}
                & 0x20 & --- & Reserved \\ \cline{2-4}
                & 0x40 & --- & Reserved \\ \cline{2-4}
                & 0x80 & 0x21 & SD Card Inserted \\ \hline
        \end{tabular}
    \end{center}
    \caption{F256 Interrupt Assignments}
\end{table}

\subsection*{sys\_int\_enable\_all -- 0x000404}
This function enables all maskable interrupts at the CPU level. It returns a system-dependent code that represents the previous level of interrupt masking.
Returns a machine dependent representation of the CPU interrupt mask state that can be used to restore the state later.
Note: this does not change the mask status of interrupts in the machine's interrupt controller, it just changes if the CPU ignores IRQs or not.

\bigskip

\begin{tabular}{|l||l|} \hline
Prototype & \lstinline!short sys_int_enable_all()! \\ \hline
\end{tabular}

\subsubsection*{Example: C}
\begin{lstlisting}
// Enable processing of IRQs
short state = sys_int_enable_all();
\end{lstlisting}

\subsubsection*{Example: Assembler}
\begin{verbatim}
; Enable processing of IRQs
jsr sys_int_enable_all
\end{verbatim}

\subsection*{sys\_int\_disable\_all -- 0x000408}
This function disables all maskable interrupts at the CPU level. It returns a system-dependent code that represents the previous level of interrupt masking. 
Returns a machine dependent representation of the CPU interrupt mask state that can be used to restore the state later.
Note: this does not change the mask status of interrupts in the machine's interrupt controller, it just changes if the CPU ignores IRQs or not.

\bigskip

\begin{tabular}{|l||l|} \hline
Prototype & \lstinline!short sys_int_disable_all()! \\ \hline
\end{tabular}

\subsubsection*{Example: C}
\begin{lstlisting}
// Disable processing of IRQs
short state = sys_int_disable_all();
\end{lstlisting}

\subsubsection*{Example: Assembler}
\begin{verbatim}
; Disable processing of IRQs
jsr sys_int_disable_all
\end{verbatim}


\subsection*{sys\_int\_restore\_all -- 0x00040C}
Restores 

\bigskip

\begin{tabular}{|l||l|} \hline
Prototype & \lstinline!void sys_int_restore_all(short state)! \\ \hline
state & the machine dependent CPU interrupt state to restore
\end{tabular}

\subsubsection*{Example: C}
\begin{lstlisting}
// Restore state of IRQ processing after enabling/disabling all
short state = ...;

sys_int_restore_all(state);
\end{lstlisting}

\subsubsection*{Example: Assembler}
\begin{verbatim}
; Restore state of IRQ processing after enabling/disabling all
move.w state,d0
jsr sys_int_restore_all
\end{verbatim}

\subsection*{sys\_int\_disable -- 0x000410}
This function disables a particular interrupt at the level of the interrupt controller. The argument passed is the number of the interrupt to disable.

\bigskip

\begin{tabular}{|l||l|} \hline
Prototype & \lstinline!void sys_int_disable(unsigned short n)! \\ \hline
n & the number of the interrupt: n[7..4] = group number, n[3..0] = individual number. \\ \hline
\end{tabular}

\subsubsection*{Example: C}
\begin{lstlisting}
// Disable the start-of-frame interrupt
sys_int_disable(INT_SOF_A);
\end{lstlisting}

\subsubsection*{Example: Assembler}
\begin{verbatim}
    move.w #INT_SOF_A,d0        ; Enable the start-of-frame interrupt
    jsr sys_int_disable
\end{verbatim}

\subsection*{sys\_int\_enable -- 0x000414}
This function enables a particular interrupt at the level of the interrupt controller.
The argument passed is the number of the interrupt to enable. Note that interrupts that are enabled at this level will still be disabled,
if interrupts are disabled globally by \verb+sys_int_disable_all+.

\bigskip

\begin{tabular}{|l||l|} \hline
Prototype & \lstinline!void sys_int_enable(unsigned short n)! \\ \hline
n & the number of the interrupt \\ \hline
\end{tabular}

\subsubsection*{Example: C}
\begin{lstlisting}
// Enable the start-of-frame interrupt
sys_int_enable(INT_SOF_A);
\end{lstlisting}

\subsubsection*{Example: Assembler}
\begin{verbatim}
    move.w #INT_SOF_A,d0        ; Enable the start-of-frame interrupt
    jsr sys_int_enable
\end{verbatim}

\subsection*{sys\_int\_register -- 0x000418}
Registers a function as an interrupt handler. An interrupt handler is a function which takes and returns no arguments and will be
run in at an elevated privilege level during the interrupt handling cycle.

The first argument is the number of the interrupt to handle, the second argument is a pointer to the interrupt handler to register.
Registering a null pointer as an interrupt handler will ``deregister'' the old handler.

The function returns the handler that was previously registered.

\begin{tabular}{|l||l|} \hline
Prototype & \lstinline!p_int_handler sys_int_register(unsigned short n, p_int_handler handler)! \\ \hline
n & the number of the interrupt \\ \hline
handler & pointer to the interrupt handler to register \\ \hline
Returns & the pointer to the previous interrupt handler \\ \hline
\end{tabular}

\subsubsection*{Example: C}
\begin{lstlisting}
// Handler for the start-of-frame interrupt
// Must be a far sub-routine (returns through RTL)
__attribute__((far)) void sof_handler() {
    // Interrupt handler code here...
}

// Register a handler for the start-of-frame interrupt
p_int_handler old = sys_int_register(INT_SOF_A, sof_handler);
\end{lstlisting}

\subsubsection*{Example: Assembler}
\begin{verbatim}
    ; Code to register the handler...
    move.l #sof_handler,-(a7)   ; push pointer to sof_handler
    move.w #INT_SOF_A,d0        ; A = the number for the SOF_A interrupt
    
    jsr sys_int_register

    addq.l #4,a7                ; Clean up the stack

    move.l d0,old               ; Save the pointer to the old handler

    ; ...

; Handler for the start-of-frame interrupt
sof_handler:
    ; Handler code here...
    rts
\end{verbatim}

\subsection*{sys\_int\_pending -- 0x00041C}
Query an interrupt to see if it is pending in the interrupt controller.
NOTE: User programs will probably never need to use this call, since it is handled by the Toolbox itself.

\bigskip

\begin{tabular}{|l||l|} \hline
Prototype & \lstinline!short sys_int_pending(unsigned short n)! \\ \hline
n & the number of the interrupt: n[7..4] = group number, n[3..0] = individual number. \\ \hline
Returns & non-zero if interrupt n is pending, 0 if not \\ \hline
\end{tabular}

\subsubsection*{Example: C}
\begin{lstlisting}
// Check to see if start-of-frame interrupt is pending
short is_pending = sys_int_pending(INT_SOF_A);
if (is_pending) {
    // The interrupt has not yet been acknowledged
}
\end{lstlisting}

\subsubsection*{Example: Assembler}
\begin{verbatim}
    ; Check to see if the start-of-frame interrupt is pending
    move.w #INT_SOF_A,d0

    jsr sys_int_pending
    
    cmp #0,d0
    beq sof_not_pending

    ; Code for when start-of-frame is pending

sof_not_pending:
\end{verbatim}

\subsection*{sys\_int\_clear -- 0x000420}
This function acknowledges the processing of an interrupt by clearing its pending flag in the interrupt controller.
NOTE: User programs will probably never need to use this call, since it is handled by the Toolbox itself.

\bigskip

\begin{tabular}{|l||l|} \hline
Prototype & \lstinline!void sys_int_clear(unsigned short n)! \\ \hline
n & the number of the interrupt: n[7..4] = group number, n[3..0] = individual number. \\ \hline
\end{tabular}

\subsubsection*{Example: C}
\begin{lstlisting}
// Acknowledge the processing of the start-of-frame interrupt
sys_int_clear(INT_SOF_A);
\end{lstlisting}

\subsubsection*{Example: Assembler}
\begin{verbatim}
    ; Acknowledge the processing of the start-of-frame interrupt
    move.w #INT_SOF_A,d0
    jsr sys_int_clear
\end{verbatim}

\subsection*{sys\_kbd\_handle\_irq -- 0x000520}
This function causes the keyboard processing code to try to process keystrokes.
In the case of the mechanical keyboard on the F256k, it will scan the keyboard matrix and process any changes to the key positions.
In the case of the optical keyboard on the F256k2, it will check the optical keyboard queue for any pending keystrokes.
In the case of the PS/2 keyboard on the F256jr, it will check the PS/2 device for pending bytes.

Internally, this function is called during the SOF interrupt on the F256k and F256k2 and is called in response to a PS/2 interrupt on the F256jr. The routine is exposed through the jumptable in case a program wants to take over the interrupt processing but still wants the Toolbox to interpret keystrokes. To get characters from the console device or to get keyboard scancodes, this function must be called periodically, as it is this function that interprets keypresses and queues up scancodes and console bytes.

\bigskip

\begin{table}[!h]\begin{tabular}{|l||l|} \hline
Prototype & \lstinline!void sys_kbd_handle_irq()! \\ \hline
Purpose & Handle an IRQ to query the keyboard \\ \hline
\end{tabular}\end{table}

\subsubsection*{Example: C}
\begin{lstlisting}
    // Look for and process pending keystrokes
    sys_kbd_handle_irq();
\end{lstlisting}

\subsubsection*{Example: Assembler}
\begin{verbatim}
    ; Look for and process pending keystrokes
    jsr sys_kbd_handle_irq
\end{verbatim}
\section{IEC Bus Functions}
This collection of functions expose low-level access to the IEC bus (aka Commodore serial bus).
The functions allow a caller to issue {\sc talk}, {\sc untalk}, {\sc listen}, {\sc unlisten} commands to devices on the bus.
The functions also allow a caller to send and receive data bytes and to check for the end-of-interaction byte (``EOI'') when
listening to a talking device.

Currently, these functions do not provide higher level access to devices, like managing files and directories on disk drives.
That level of access may be added in the future.
Similarly, only the original, slow protocol of data transfer is supported.
Faster protocols like JiffyDOS or the C128 faster data transfer protocols are not yet supported.

As of version 1.01, these functions should be considered experimental and quite probably buggy.

\subsection*{sys\_iecll\_ioinit -- 0xFFE130 -- v1.01}
This function initializes the IEC port to make sure the pins are in a valid start condition. It should be called before first accessing the IEC port.

\bigskip

\begin{table}[!h]\begin{tabular}{|l||l|} \hline
Prototype & \lstinline!short sys_iecll_ioinit()! \\ \hline
Purpose & Initialize the IEC interface \\ \hline
Returns & short 0 on success, -1 if no devices found \\ \hline
\end{tabular}\end{table}

\subsubsection*{Example: C}
\begin{lstlisting}
// Initialize the IEC port
sys_iecll_ioinit();
\end{lstlisting}

\subsubsection*{Example: Assembler}
\begin{verbatim}
; Initialize the IEC port
jsl sys_iecll_ioinit
\end{verbatim}


\subsection*{sys\_iecll\_talk -- 0xFFE140 -- v1.01}
Send a {\sc talk} command to a device on the IEC bus.

\bigskip

\begin{table}[!h]\begin{tabular}{|l||l|} \hline
Prototype & \lstinline!short sys_iecll_talk(uint8_t device)! \\ \hline
Purpose & Send a {\sc talk} command to a device \\ \hline
device & the number of the device to become the talker \\ \hline
Returns & short \\ \hline
\end{tabular}\end{table}

\subsubsection*{Example: C}
\begin{lstlisting}
// Tell IEC device #8 to start talking
sys_iecll_talk(8);
\end{lstlisting}

\subsubsection*{Example: Assembler}
\begin{verbatim}
    ; Tell IEC device #8 to start talking
    lda #8
    jsl sys_iecll_talk
\end{verbatim}


\subsection*{sys\_iecll\_talk\_sa -- 0xFFE144 -- v1.01}
Send the secondary address for the {\sc talk} command.
Usually, this specifies the channel on the IEC device.

\bigskip

\begin{table}[!h]\begin{tabular}{|l||l|} \hline
Prototype & \lstinline!short sys_iecll_talk_sa(uint8_t secondary_address)! \\ \hline
Purpose & Send the secondary address to the {\sc talk} command, release ATN,
and turn around control of the bus \\ \hline
secondary\_address & the secondary address to send \\ \hline
Returns & short \\ \hline
\end{tabular}\end{table}

\subsubsection*{Example: C}
\begin{lstlisting}
// Set the talk IEC channel to 2
sys_iecll_talk_sa(2);
\end{lstlisting}

\subsubsection*{Example: Assembler}
\begin{verbatim}
    ; Set the talk IEC channel to 2
    lda #2
    jsl sys_iecll_talk_sa
\end{verbatim}


\subsection*{sys\_iecll\_listen -- 0xFFE14C -- v1.01}
Send the {\sc listen} command to a device on the IEC bus.

\bigskip

\begin{table}[!h]\begin{tabular}{|l||l|} \hline
Prototype & \lstinline!short sys_iecll_listen(uint8_t device)! \\ \hline
Purpose & Send a {\sc listen} command to a device \\ \hline
device &  \\ \hline
Returns & short the number of the device to become the listener \\ \hline
\end{tabular}\end{table}

\subsubsection*{Example: C}
\begin{lstlisting}
// Tell IEC device #8 to start listening
sys_iecll_listen(8);
\end{lstlisting}

\subsubsection*{Example: Assembler}
\begin{verbatim}
    ; Tell IEC device #8 to start listening
    lda #8
    jsl sys_iecll_listen
\end{verbatim}


\subsection*{sys\_iecll\_listen\_sa -- 0xFFE150 -- v1.01}
\begin{table}[!h]\begin{tabular}{|l||l|} \hline
Prototype & \lstinline!short sys_iecll_listen_sa(uint8_t secondary_address)! \\ \hline
Purpose & Send the secondary address to the {\sc listen} command and release ATN \\ \hline
secondary\_address & the secondary address to send \\ \hline
Returns & short \\ \hline
\end{tabular}\end{table}

\subsubsection*{Example: C}
\begin{lstlisting}
// Set the listen IEC channel to 2
sys_iecll_listen_sa(2);
\end{lstlisting}

\subsubsection*{Example: Assembler}
\begin{verbatim}
    ; Set the listen IEC channel to 2
    lda #2
    jsl sys_iecll_listen_sa
\end{verbatim}


\subsection*{sys\_iecll\_untalk -- 0xFFE148 -- v1.01}
Tell the device currently talking on the IEC bus to stop talking.

\bigskip

\begin{table}[!h]\begin{tabular}{|l||l|} \hline
Prototype & \lstinline!void sys_iecll_untalk()! \\ \hline
Purpose & Send the {\sc untalk} command to all devices and drop ATN \\ \hline
\end{tabular}\end{table}

\subsubsection*{Example: C}
\begin{lstlisting}
// Send the UNTALK command
sys_iecll_untalk();
\end{lstlisting}

\subsubsection*{Example: Assembler}
\begin{verbatim}
    ; Send the UNTALK command
    jsl sys_iecll_untalk
\end{verbatim}


\subsection*{sys\_iecll\_unlisten -- 0xFFE154 -- v1.01}
Tell all devices listening on the IEC bus to stop listening.

\bigskip

\begin{table}[!h]\begin{tabular}{|l||l|} \hline
Prototype & \lstinline!void sys_iecll_unlisten()! \\ \hline
Purpose & Send the {\sc unlisten} command to all devices \\ \hline
\end{tabular}\end{table}

\subsubsection*{Example: C}
\begin{lstlisting}
// Send the UNLISTEN command
sys_iecll_unlisten();
\end{lstlisting}

\subsubsection*{Example: Assembler}
\begin{verbatim}
    ; Send the UNLISTEN command
    jsl sys_iecll_unlisten
\end{verbatim}

\subsection*{sys\_iecll\_in -- 0xFFE134 -- v1.01}
Wait for and receive a byte from the current talker on the IEC bus.

NOTE: this transfer uses the default, low speed protocol of the IEC bus.
Currently, higher speed protocols like JiffyDOS and C128 mode are not supported.

\bigskip

\begin{table}[!h]\begin{tabular}{|l||l|} \hline
Prototype & \lstinline!uint8_t sys_iecll_in()! \\ \hline
Purpose & Try to get a byte from the IEC bus \\ \hline
Returns & uint8\_t the byte read \\ \hline
\end{tabular}\end{table}

\subsubsection*{Example: C}
\begin{lstlisting}
// Receive a byte from the current IEC bus talking device
uint8_t data = sys_iecll_in();
\end{lstlisting}

\subsubsection*{Example: Assembler}
\begin{verbatim}
    ; Receive a byte from the current IEC bus talking device
    jsl sys_iecll_in
    ; A[0..8] contains the byte
\end{verbatim}


\subsection*{sys\_iecll\_eoi -- 0xFFE138 -- v1.01}
Check to see if the data byte received from the IEC bus using \lstinline|sys_iecll_in| was the
last byte the talker is going to transmit (the end-of-interaction or EOI byte).

\bigskip

\begin{table}[!h]\begin{tabular}{|l||l|} \hline
Prototype & \lstinline!short sys_iecll_eoi()! \\ \hline
Purpose & Check to see if the last byte read was an EOI byte \\ \hline
Returns & short 0 if not EOI, any other number if EOI \\ \hline
\end{tabular}\end{table}

\subsubsection*{Example: C}
\begin{lstlisting}
// Read data bytes from the talker until we get the EOI flag
do {
    uint8_t data = sys_iecll_in();

    // Do something with the data byte...
} while(!sys_iecll_eoi());
\end{lstlisting}

\subsubsection*{Example: Assembler}
\begin{verbatim}
read_in:
    jsl sys_iecll_in    ; Get a byte from the IEC talker

    ; Do somethinig with the data byte...

    ; Check to see if we got the last byte from the talker
    jsl sys_iecll_eoi
    beq read_in         ; No... keep reading bytes...=
\end{verbatim}


\subsection*{sys\_iecll\_out -- 0xFFE13C -- v1.01}
Send a byte to all the devices listening on the IEC bus.

NOTE: this transfer uses the default, low speed protocol of the IEC bus.
Currently, higher speed protocols like JiffyDOS and C128 mode are not supported.

\bigskip

\begin{table}[!h]\begin{tabular}{|l||l|} \hline
Prototype & \lstinline!void sys_iecll_out(uint8_t byte)! \\ \hline
Purpose & Send a byte to the IEC bus. Actually sends the previous byte and queues the current byte. \\ \hline
byte & the byte to send \\ \hline
\end{tabular}\end{table}

\subsubsection*{Example: C}
\begin{lstlisting}
// Send a byte to all the listeners on the IEC bus
uint8_t data = 0x41;
sys_iecll_out(data);
\end{lstlisting}

\subsubsection*{Example: Assembler}
\begin{verbatim}
    ; Send a byte to all the listeners on the IEC bus
    lda #$41
    jsl sys_iecll_out
\end{verbatim}


\subsection*{sys\_iecll\_reset -- 0xFFE158 -- v1.01}
Toggle the RESET line on the IEC bus. This should trigger all IEC bus devices to reset.

\bigskip

\begin{table}[!h]\begin{tabular}{|l||l|} \hline
Prototype & \lstinline!void sys_iecll_reset()! \\ \hline
Purpose & Assert and release the reset line on the IEC bus \\ \hline
\end{tabular}\end{table}

\subsubsection*{Example: C}
\begin{lstlisting}
// Reset all devices on the IEC bus
sys_iecll_reset();
\end{lstlisting}

\subsubsection*{Example: Assembler}
\begin{verbatim}
    ; Reset all devices on the IEC bus
    jsl sys_iecll_reset
\end{verbatim}
