\chapter{Appendix}

\section{Console IOCTRL Commands}
The console channel driver supports the following commands for \verb+sys_chan_ioctrl+.
None of these IOCTRL commands require a buffer, so passing NULL for the buffer and 0 for the size is recommended.

\begin{enumerate}
	\item \verb+CON_IOCTRL_ANSI_ON+: Turn on ANSI escape sequence processing.\footnote{ANSI processing is on by default.}
	\item \verb+CON_IOCTRL_ANSI_OFF+: Turn off ANSI escape sequence processing.
	\item \verb+CON_IOCTRL_ECHO_ON+: Turn on character echoing for \verb+sys_chan_read_b+.\footnote{Echo is on be default.}
	\item \verb+CON_IOCTRL_ECHO_OFF+: Turn off character echoing for \verb+sys_chan_read_b+.
	\item \verb+CON_IOCTRL_BREAK+: Check to see if the user has pressed the BREAK key sequence.\footnote{{\tt sys\_chan\_ioctrl} will return a non-zero value if the BREAK key was pressed, and 0 if not. On all Foenix machines, CTRL-C (code point 0x03) will be treated as the BREAK key. On the A2560K, the combination Foenix-ESC will also work as the BREAK key. On the F256K, the RUN/STOP key will be treated as the BREAK key.}
	\item \verb+CON_IOCTRL_CURS_ON+: The text mode cursor should be visible.\footnote{Cursor is on by default.}
	\item \verb+CON_IOCTRL_CURS_OFF+: The text mode cursor should be hidden
\end{enumerate}

\section{ANSI Terminal Codes}
Foenix Toolbox supports a basic subset of the VT102 ANSI terminal codes. The following escape sequences are shown in table~\ref{tbl:ansi_terminal_codes}.

\begin{table}
	\begin{center}
		\begin{tabular}{|l|l|l|} \hline
		Sequence & Name & Function \\ \hline\hline
		\verb+ESC [ # @+ & ICH & Insert characters \\ \hline
		\verb+ESC [ # A+ & CUU & Move the cursor up \\ \hline
		\verb+ESC [ # B+ & CUF & Move the cursor forward \\ \hline
		\verb+ESC [ # C+ & CUB & Move the cursor back \\ \hline
		\verb+ESC [ # D+ & CUD & Move the cursor down \\ \hline
		\verb+ESC [ # J+ & ED & Erase the screen \\ \hline
		\verb+ESC [ # K+ & EL & Erase the line \\ \hline
		\verb+ESC [ # P+ & DCH & Delete characters \\ \hline
		\verb+ESC [ # ; # H+ & CUP & Set the cursor position \\ \hline
		\verb+ESC [ # m+ & SGR & Set the graphics rendition \\ \hline
		\end{tabular}
	\end{center}
    \caption{ANSI Terminal Codes}
    \label{tbl:ansi_terminal_codes}
\end{table}

For the SGR sequence, a fairly limited set of codes are currently supported, mainly to do with the color and intensity of the text (see table:~\ref{tbl:ansi_sgr_codes}).
\begin{table}
	\begin{center}
		\begin{tabular}{|l|l|l|} \hline
		Code & Function \\ \hline\hline
		0 & Reset to defaults \\ \hline
		1 & High intensity / Bold \\ \hline
		2 & Low intensity / Normal \\ \hline
		22 & Normal \\ \hline
		30 -- 37 & Set foreground color \\ \hline
		40 -- 47 & Set background color \\ \hline
		90 -- 97 & Set bright foreground color \\ \hline
		100 -- 107 & Set bright background color \\ \hline
		\end{tabular}
	\end{center}
    \caption{ANSI SGR Codes}
    \label{tbl:ansi_sgr_codes}
\end{table}

NOTE: If the program does not want the console to interpret ANSI codes, this feature can be turned off by calling \verb+sys_chan_ioctrl+ on the console channel to be changed. A command of 0x01 will turn ANSI interpretation on, while a command of 0x02 will turn it off. When ANSI interpretation is turned off, only the core ASCII control characters will still be recognized: 0x08 (backspace), 0x09 (TAB), 0x0A (linefeed), and 0x13 (carriage return).

For key presses, escape codes (see table~\ref{tbl:ansi_key_codes}) are sent to the calling program,
when one of the \verb+sys_chan_read+ functions is used on the channel.
Note that this feature is always on in the current system. Also, in the following codes, there are no actual spaces.

\begin{table}
	\begin{center}
		\begin{tabular}{|l|l|l|} \hline
		Key & Code \\ \hline\hline
		ESC & \verb+ESC ESC+ \\ \hline
		Cursor UP & \verb+ESC [ # A+ \\ \hline
		Cursor Down & \verb+ESC [ # B+ \\ \hline
		Cursor Left & \verb+ESC [ # C+ \\ \hline
		Cursor Right & \verb+ESC [ # D+ \\ \hline
		HOME & \verb+ESC [ 1 ; # ~+ \\ \hline
		INS & \verb+ESC [ 2 ; # ~+ \\ \hline
		DELETE & \verb+ESC [ 3 ; # ~+ \\ \hline
		END & \verb+ESC [ 4 ; # ~+ \\ \hline
		PAGE UP & \verb+ESC [ 5 ; # ~+ \\ \hline
		PAGE DOWN & \verb+ESC [ 8 ; # ~+ \\ \hline
		F1 & \verb+ESC [ 11 ; # ~+ \\ \hline
		F2 & \verb+ESC [ 12 ; # ~+ \\ \hline
		F3 & \verb+ESC [ 13 ; # ~+ \\ \hline
		F4 & \verb+ESC [ 14 ; # ~+ \\ \hline
		F5 & \verb+ESC [ 15 ; # ~+ \\ \hline
		F6 & \verb+ESC [ 17 ; # ~+ \\ \hline
		F7 & \verb+ESC [ 18 ; # ~+ \\ \hline
		F8 & \verb+ESC [ 19 ; # ~+ \\ \hline
		F9 & \verb+ESC [ 20 ; # ~+ \\ \hline
		F10 & \verb+ESC [ 21 ; # ~+ \\ \hline
		F11 & \verb+ESC [ 23 ; # ~+ \\ \hline
		F12 & \verb+ESC [ 24 ; # ~+ \\ \hline
		\end{tabular}
	\end{center}
    \caption{ANSI Key Codes}
    \label{tbl:ansi_key_codes}
\end{table}

% Key
% Code


% The “#” in the sequences above represent an optional modifier code. If SHIFT, CTRL, or ALT is pressed with the key, the number sign is replaced with a decimal number representing a bitfield of the modifier keys, followed by a semicolon. The bit values are: SHIFT = 1, ALT = 2, CTRL = 4, and OS (Foenix) = 8. 

\section{Keyboard Scan codes}
Foenix Toolbox uses the same Foenix scan codes that the original 65816 Foenix kernel and Foenix/MCP used.
These scan codes are derived from the standard ``set 1'' scan codes with modifications to get the scan codes to fit within a single byte.
The base scan codes for a US QWERTY keyboard are listed below.

When a key is pressed or released, bits 0 -- 6 are the same, and follow the table below.
A ``make'' scan code is sent when the key is pressed.
For make scan codes, bit 7 is clear (0).
A ``break'' scan code is sent when a key is released.
For break scan codes, bit 7 is set (1).

Example---the user presses and releases the space bar: Two scan codes will be sent.
First, the make code 0x39 will be sent.
Second, the break scan code of 0xB9 will be sent when the key is released.

\begin{table}[ht]
    \begin{center}
        \begin{tabular}{|l||l|l|l|l|l|l|l|} \hline
                 & 0x00     & 0x10      & 0x20      & 0x30     & 0x40   & 0x50   & 0x60    \\ \hline\hline
            0x00 &          &    Q      &    D      &    B     &   F6   &  KP2   & PRSN    \\ \hline
            0x01 & ESC      &    W      &    F      &    N     &   F7   &  KP3   & PAUSE   \\ \hline
            0x02 & 1	    &    E      & G         & M        & F8     &  KP0   & INS     \\ \hline
            0x03 & 2	    &    R      & H         & \verb+<+ & F9     &  KP.   & HOME    \\ \hline
            0x04 & 3	    &    T      & J         & \verb+>+ & F10    &        & PGUP    \\ \hline
            0x05 & 4	    &    Y      & K         & \verb+/+ & NUMLCK &        & DEL     \\ \hline
            0x06 & 5	    &    U      & L         & R SHFT   & SCRLCK &        & END     \\ \hline
            0x07 & 6	    &    I      & \verb+;+  & KP*      & KP7    & F11    & PGDN    \\ \hline
            0x08 & 7	    &    O      & \verb+"+  & L ALT    & KP8    & F12    & UP      \\ \hline
            0x09 & 8        &    P      & \verb+~+  & SPACE    & KP9    &        & LEFT    \\ \hline
            0x0A & 9        & \verb+[+  & L SHFT    & CAPS     & KP-    &        & DOWN    \\ \hline
            0x0B & 0        & \verb+]+  & \verb+\+  & F1       & KP4    & L FNX  & RIGHT   \\ \hline
            0x0C & \verb+-+ & ENTER     & Z         & F2       & KP5    & R ALT  & KP/     \\ \hline
            0x0D & \verb+=+ & L CTRL    & X         & F3       & KP6    & R FNX  & KPENTER \\ \hline
            0x0E & BKSP     & A         & C         & F4       & KP+    & R CTRL &         \\ \hline
            0x0F & TAB      & S         & V         & F5       & KP1    &        &         \\ \hline
        \end{tabular}
    \end{center}
    \caption{Keyboard Scan codes}
    \label{tbl:kbd_scancodes}
\end{table}

\section{Useful Data Structures}

\subsection*{Time}
\begin{lstlisting}
// Structure used for real time clock functions
struct s_time {
    short year;       // Year (0 - 9999)
    short month;      // Month (1 = January through 12 = December)
    short day;        // Day of month (1 - 31)
    short hour;       // Hour (0 - 12 / 23)
    short minute;     // Minute (0 - 59)
    short second;     // Seconds (0 - 59)
    short is_pm;	    // For 12-hour clock, 1 = PM
    short is_24hours; // 1 = clock is 24-hours, 0 = clock is 12-hours
}
\end{lstlisting}

\subsection*{Directory Entries}
\begin{lstlisting}
// Structure used for directory entry information
struct s_file_info {
    long size;                // Size of the file in bytes
    unsigned short date;      // Creation date
    unsigned short time;      // Creation time
    unsigned char attributes; // Attribute bits
    char name[MAX_PATH_LEN];  // Name of the file (256 bytes)
}
\end{lstlisting}

File attribute bits:

\bigskip

\begin{tabular}{|l|l|} \hline
0x01 & Read only \\ \hline
0x02 & Hidden file \\ \hline
0x04 & System file \\ \hline
0x10 & Directory \\ \hline
0x20 & Archive \\ \hline
\end{tabular}

\subsection*{System Information}
\begin{lstlisting}
/*
 * Structure to describe the hardware
 */
struct s_sys_info {
    uint16_t mcp_version;     		/* Current version of the MCP kernel */
    uint16_t mcp_rev;         		/* Current revision, or sub-version of the MCP kernel */
    uint16_t mcp_build;       		/* Current vuild # of the MCP kernel */
    uint16_t model;           		/* Code to say what model of machine this is */
    uint16_t sub_model;         	/* 0x00 = PB, 0x01 = LB, 0x02 = CUBE */
    const char * model_name;        /* Human readable name of the model of the computer */
    uint16_t cpu;             		/* Code to say which CPU is running */
    const char * cpu_name;          /* Human readable name for the CPU */
    uint32_t cpu_clock_khz;     	/* Speed of the CPU clock in KHz */
    unsigned long fpga_date;        /* YYYYMMDD */    
    uint16_t fpga_model;       		/* FPGA model number */
    uint16_t fpga_version;    		/* FPGA version */
    uint16_t fpga_subver;     		/* FPGA sub-version */
    uint32_t system_ram_size;		/* The number of bytes of system RAM on the board */
    bool has_floppy;                /* TRUE if the board has a floppy drive installed */
    bool has_hard_drive;            /* TRUE if the board has a PATA device installed */
    bool has_expansion_card;        /* TRUE if an expansion card is installed on the device */
    bool has_ethernet;              /* TRUE if an ethernet port is present */
    uint16_t screens;         		/* How many screens are on this computer */
};
\end{lstlisting}

\subsection*{Model and CPU IDs}
The numbers listed in table~\ref{tbl:fnx_model_ids} are used to distinguish between the different models of Foenix computers.
The numbers listed in table~\ref{tbl:fnx_cpu_ids} are used to distinguish between the different CPUs.
Both the machine and CPU IDs are also used by the Toolbox's make file.

\begin{table}
	\begin{center}
		\begin{tabular}{|l|l|} \hline
			Model & Number \\ \hline\hline
			C256 FMX & 0 \\ \hline
			C256 U & 1 \\ \hline
			C256 GenX & 4 \\ \hline
			C256 U+ & 5 \\ \hline
			A2560 U+ & 6 \\ \hline
			A2560 X & 7 \\ \hline
			A2560 U & 9 \\ \hline
			A2560 K & 11 \\ \hline
		\end{tabular}
	\end{center}
	\caption{Foenix Model IDs}
	\label{tbl:fnx_model_ids}
\end{table}

\begin{table}
	\begin{center}
		\begin{tabular}{|l|l|} \hline
			CPU & Number \\ \hline\hline
			M68SEC000 & 0 \\ \hline
			M68020 & 1 \\ \hline
			M68EC020 & 2 \\ \hline
			M68030 & 3 \\ \hline
			M68EC030 & 4 \\ \hline
			M68040 & 5 \\ \hline
			M68040V & 6 \\ \hline
			ME68EC040 & 7 \\ \hline
			i486DX 50 & 8 \\ \hline
			i468DX 60 & 9 \\ \hline
			i468DX4 & 10 \\ \hline
		\end{tabular}
	\end{center}
	\caption{Foenix CPU IDs}
	\label{tbl:fnx_cpu_ids}
\end{table}

\subsection*{Screen Information}
There are several structures defined to provide information about the text screen and to be used in controlling various aspects
of the text screen.

\begin{lstlisting}
/*
 * Structure to specify the size of a rectangle
 */
typedef struct s_extent {
    short width;        /**< The width of the region */
    short height;       /**< The height of the region */
} t_extent, *p_extent;

/*
 * Structure to specify the location of a point on the screen
 */
typedef struct s_point {
    short x;                /**< The column of the point */
    short y;                /**< The row of the point */
} t_point, *p_point;

/*
 * Structure to specify a rectangular area on the screen
 */
typedef struct s_rect {
    t_point origin;         /**< The upper-left corner of the rectangle */
    t_extent size;          /**< The size of the rectangle */
} t_rect, *p_rect;
\end{lstlisting}

The capabilities of the screen are listed in the text capabilities structure.
These capabilities include the supported display modes on the screen (as a bit field, the values of which are listed in table~\ref{tbl:fnx_screen_modes}),
the number and sizes of the fonts supported, and the display resolutions supported.

\begin{lstlisting}
/*
* Structure to specify the capabilities of a screen's text driver
*/
typedef struct s_txt_capabilities {
    short number;               /**< The unique ID of the screen */
    short supported_modes;      /**< The display modes supported on this screen */
    short font_size_count;      /**< The number of supported font sizes */
    p_extent font_sizes;        /**< Pointer to a list of t_extent listing all supported font sizes (in pixels) */
    short resolution_count;     /**< The number of supported display resolutions */
    p_extent resolutions;       /**< Pointer to a list of t_extent listing all supported display resolutions (in pixels) */
} t_txt_capabilities, *p_txt_capabilities;
\end{lstlisting}

\begin{table}
	\begin{center}
		\begin{tabular}{|l|l|} \hline
			Mode & Number \\ \hline\hline
			\verb+TXT_MODE_TEXT+ & 0x0001 \\ \hline
			\verb+TXT_MODE_BITMAP+ & 0x0002 \\ \hline
			\verb+TXT_MODE_SPRITE+ & 0x0004 \\ \hline
			\verb+TXT_MODE_TILE+ & 0x0008 \\ \hline
			\verb+TXT_MODE_SLEEP+ & 0x0010 \\ \hline
		\end{tabular}
	\end{center}
	\caption{Toolbox Screen Mode Flags}
	\label{tbl:fnx_screen_modes}
\end{table}

\section{Foenix Executable File Formats}

\subsection*{PGX File Format}
The PGX file format is the simplest executable format. It is similar in scale to MS-DOS's COM format, or the Commodore PRG format.
It consists of a single segment of data to be loaded to a specific address, where that address is also the starting address.
PGX starts with a header to identify the file and the starting address:

\begin{itemize}
	\item The first three bytes are the ASCII codes for ``PGX''. 
    \item The fourth byte is the CPU and version identification byte.
	Bits 0 through 3 represent the CPU code, and bits 4 through 7 represent the version of PGX supported.
	At the moment, there is just version 0. The CPU code can be 1 for the WDC65816, or 2 for the M680x0.
    \item The next four bytes (that is, bytes 4 through 7) are the address of the destination, in big-endian format (most significant byte first).
	This address is both the address of the location in which to load the first byte of the data and is also the starting address for the file.
\end{itemize}

All bytes after the header are the contents of the file to be loaded into memory.

\subsection*{PGZ File Format}
The PGZ is a more complex format that supports multiple loadable segments, but is still to be loaded in set locations in memory.
The first byte of the file is a file signature and also a version tag.

If the first byte is an upper case Z, the file is a 24-bit PGZ file (i.e. all addresses and sizes specified in the file are 24-bits).
If the file is a lower case Z, the file is a 32-bit PGZ file (all address and sizes are 32-bits in length).
Note that all addresses and sizes are in little-endian format (that is, least-significant byte first).

After the initial byte, the remainder of the PGZ file consists of segments, one after the other. Each segment consists of two or three fields,
shown in table~\ref{tbl:pgz_file_segments}.

\begin{table}
	\begin{center}
		\begin{tabular}{|l|l|l|} \hline
			Field &	Size & Description \\ \hline\hline
			\verb+address+ & 3 (“Z”) or 4 (“z”) bytes & The target address for this segment \\ \hline
			\verb+size+ & 3 (“Z”) or 4 (“z”) bytes & The number of bytes in the data field \\ \hline
			\verb+data+ & \verb+size+ bytes & The data to be loaded [optional] \\ \hline
		\end{tabular}
	\end{center}
	\caption{PGZ File Segments}
	\label{tbl:pgz_file_segments}
\end{table}

For a particular segment, if the size field is 0, there will be no bytes in the data field,
and the segment specifies the starting address of the entire program.
At least one such segment must be present in the PGZ file for it to be executable.
If more than one is present, the last one will be the one used to specify the starting address.

\section{Memory Map}
The Foenix Toolbox uses different sections of both flash and RAM memory to provide its functions.
The memory map in table~\ref{tbl:memory_map} marks the major areas and pseudo-registers for the Toolbox.
The table also marks out sections of memory that are reserved for future use by the Toolbox (that is,
sections not currently used but which may be used in the future), and areas specifically reserved for user
programs (called out in boldface).

Roughly speaking, all bank 0 RAM below 0x00:D000 and all RAM from bank 1 through bank 6 are free for user programs
to do with as they wish. The stack is allocated and shared between the user programs and the toolbox functions.
If a program wishes to move the stack for some reason, this should be safe enough to do, although some Toolbox functions
might fail, if they use too much of the stack the user program reserved.

Of course, if a program takes complete control over the F256 and does not require any toolbox functions, the program is free
to do whatever it requires with memory. This memory map is only of importance for programs that need the Toolbox functions to
work correctly.

\begin{table}
	\begin{center}
		\begin{tabular}{|l|l|} \hline
			Address & Purpose \\ \hline\hline
			{\tt 0x00:0000 - 0x00:CFFF} & {\bf User Memory} \\ \hline
			{\tt 0x00:D000 - 0x00:DFFF} & Toolbox Low Memory \\ \hline
			{\tt 0x00:E000 - 0x00:EDEA} & Reserved \\ \hline
			{\tt 0x00:EDEB - 0x00:FDEB} & Stack \\ \hline
			{\tt 0x00:FDEC - 0x00:FDEF} & User IRQ Vector \\ \hline
			{\tt 0x00:FDF4 - 0x00:FDF7} & User NMI Vector \\ \hline
			{\tt 0x00:FDF8 - 0x00:FDFF} & Reserved \\ \hline
			{\tt 0x00:FE00 - 0x00:FEFF} & Toolbox Direct Page \\ \hline
			{\tt 0x00:FF00 - 0x00:FFFF} & Toolbox Bootstrap Shadow \\ \hline
			{\tt 0x01:0000 - 0x06:FFFF} & {\bf User Memory} (F256jr, F256Ke) \\ \hline	
			{\tt 0x01:0000 - 0x0E:FFFF} & {\bf User Memory} (F256K2e) \\ \hline	
			{\tt 0x07:0000 - 0x07:FFFF} & Toolbox Memory (F256jr, F256Ke) \\ \hline
			{\tt 0x0F:0000 - 0x0F:FFFF} & Toolbox Memory (F256K2e) \\ \hline
			{\tt 0xFC:0000 - 0xFF:DFFF} & Toolbox Firmware \\ \hline
			{\tt 0xFF:E000 - 0xFF:FEFF} & Toolbox Jump Table \\ \hline
			{\tt 0xFF:FF00 - 0xFF:FFFF} & Toolbox Bootstrap and Vectors \\ \hline
		\end{tabular}
	\end{center}
	\caption{Toolbox Memory Usage}
	\label{tbl:memory_map}
\end{table}