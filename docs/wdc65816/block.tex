\section{Block Device Functions}
The block device functions provide low-level support for access to block-based storage devices.
The main operations on block devices are reading a block of data from a device (given the device number and the
address of the block to read), and writing a block of data to the device.
These functions are used by the driver to the FatFS library to provide FAT32 file based access to those
block devices.
Currently, for the F256, this support is limited to SD cards. The F256jr and F256K have just the one SD card,
but the F256K2e has an internal and an external SD card.
Future machines might provide additional block devices ({\it e.g.} floppy drives or hard drives), and if someone
wanted to build some sort of block device for the F256, a driver to support it could be added to the Toolbox to
add those devices to the FAT32 file support.

\subsection*{sys\_bdev\_register -- 0xFFE060}
Register a device driver for a block device. A device driver consists of a structure that specifies the name and number of the device as well as the various handler functions that implement the block device calls for that device.

See the section ``Extending the System'' below for more information.

\bigskip

\begin{tabular}{|l||l|} \hline
Prototype & \lstinline!short sys_bdev_register(p_dev_block device)! \\ \hline
device & pointer to the description of the device to register \\ \hline
Returns & 0 on success, negative number on error \\ \hline
\end{tabular}

\subsection*{sys\_bdev\_read -- 0xFFE064}
Read a block from a block device. Returns the number of bytes read.

\bigskip

\begin{tabular}{|l||l|} \hline
Prototype & \lstinline!short sys_bdev_read(short dev, long lba, uint8_t * buffer, short size)! \\ \hline
dev & the number of the device \\ \hline
lba & the logical block address of the block to read \\ \hline
buffer & the buffer into which to copy the block data \\ \hline
size & the size of the buffer. \\ \hline
Returns & number of bytes read, any negative number is an error code \\ \hline
\end{tabular}

\subsubsection*{Example: C}
\begin{lstlisting}
    unsigned char buffer[512];
    
    // Read the MBR of the interal SD card
    short n = sys_bdev_read(BDEV_SD1, 0, buffer, 512);
\end{lstlisting}

\subsubsection*{Example: Assembler}
\begin{verbatim}
    pei #512			; Push the size of the buffer
    pei #`buffer		; Push the pointer to the read buffer
    pei #<>buffer
    pei #0				; Push LBA = 0
    pei #0
    lda #1				; The device number for the internal SD

    jsl sys_bdev_read	; Read sector 0

    ply					; Clean up the stack
    ply
    ply
    ply
    ply
\end{verbatim}

\subsection*{sys\_bdev\_write -- 0xFFE068}
Write a block from a block device. Returns the number of bytes written.

\bigskip

\begin{tabular}{|l||l|} \hline
Prototype & \lstinline!short sys_bdev_write(short dev, long lba, const uint8_t * buffer, short size)! \\ \hline
dev & the number of the device \\ \hline
lba & the logical block address of the block to write \\ \hline
buffer & the buffer containing the data to write \\ \hline
size & the size of the buffer. \\ \hline
Returns & number of bytes written, any negative number is an error code \\ \hline
\end{tabular}

\subsubsection*{Example: C}
\begin{lstlisting}
    unsigned char buffer[512];

    // Fill in the buffer with data...
    
    // Write the MBR of the interal SD card
    short n = sys_bdev_write(BDEV_SD1, 0, buffer, 512);
\end{lstlisting}

\subsubsection*{Example: Assembler}
\begin{verbatim}
    pei #512			; Push the size of the buffer
    pei #`buffer		; Push the pointer to the read buffer
    pei #<>buffer
    pei #0				; Push LBA = 0
    pei #0
    lda #1				; The device number for the internal SD

    jsl sys_bdev_write	; Write sector 0

    ply					; Clean up the stack
    ply
    ply
    ply
    ply
\end{verbatim}

\subsection*{sys\_bdev\_status -- 0xFFE06C}
Gets the status of a block device. The meaning of the status bits is device specific, but there are two bits that are required in order to support the file system:
\begin{itemize}
    \item 0x01: Device has not been initialized yet
    \item 0x02: Device is present
\end{itemize}

\bigskip

\begin{tabular}{|l||l|} \hline
Prototype & \lstinline!short sys_bdev_status(short dev)! \\ \hline
dev & the number of the device \\ \hline
Returns & the status of the device \\ \hline
\end{tabular}

\subsubsection*{Example: C}
\begin{lstlisting}
    short bdev = ...; // The device number
    short status = sys_bdev_status(bdev);
\end{lstlisting}

\subsubsection*{Example: Assembler}
\begin{verbatim}
    lda bdev            ; Load the device number
    jsl sys_bdev_flush  ; Attempt to flush pending writes

    ; Status is in the accumulator
\end{verbatim}


\subsection*{sys\_bdev\_flush -- 0xFFE070}
Ensure any pending writes to a block device are completed.

\bigskip

\begin{tabular}{|l||l|} \hline
Prototype & \lstinline!short sys_bdev_flush(short dev)! \\ \hline
dev & the number of the device \\ \hline
Returns & 0 on success, any negative number is an error code \\ \hline
\end{tabular}

\subsubsection*{Example: C}
\begin{lstlisting}
    short bdev = ...; // The device number
    sys_bdev_flush(bdev);
\end{lstlisting}

\subsubsection*{Example: Assembler}
\begin{verbatim}
    lda bdev            ; Load the device number
    jsl sys_bdev_flush  ; Attempt to flush pending writes
\end{verbatim}

\subsection*{sys\_bdev\_ioctrl -- 0xFFE074}
Send a command to a block device. The mapping of commands and their actions are device-specific. The return value is also device and command-specific.

Four commands should be supported by all devices:
\begin{itemize}
    \item \verb+GET_SECTOR_COUNT+ (1): Returns the number of physical sectors on the device
    \item \verb+GET_SECTOR_SIZE+ (2): Returns the size of a physical sector in bytes
    \item \verb+GET_BLOCK_SIZE+ (3): Returns the block size of the device. Really only relevant for flash devices and only needed by FatFS
    \item \verb+GET_DRIVE_INFO+ (4): Returns the identification of the drive
\end{itemize}

\bigskip

\begin{tabular}{|l||l|} \hline
Prototype & \lstinline!short sys_bdev_ioctrl(short dev, short command, uint8_t * buffer, short size)! \\ \hline
dev & the number of the device \\ \hline
command & the number of the command to send \\ \hline
buffer & pointer to bytes of additional data for the command \\ \hline
size & the size of the buffer \\ \hline
Returns & 0 on success, any negative number is an error code \\ \hline
\end{tabular}

\subsubsection*{Example: C}
\begin{lstlisting}
    short dev = ...;                // The device number
    short cmd = ...;                // The command
    short r = sys_bdev_ioctrl(dev, cmd, 0, 0); // Send simple command
\end{lstlisting}

\subsubsection*{Example: Assembler}
\begin{verbatim}
    pei #0                  ; Push buffer size of 0
    pei #0                  ; Push null pointer for buffer
    pei #0
    lda cmd                 ; Push the command number
    pha
    lda bdev                ; Select the block device
    
    jsl sys_bdev_ioctrl     ; Send the command

    ply                     ; Clean up the stack
    ply
    ply
\end{verbatim}